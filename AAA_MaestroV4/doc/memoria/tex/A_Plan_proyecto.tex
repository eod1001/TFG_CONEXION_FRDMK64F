\apendice{Plan de Proyecto Software}\label{ch:plan}

\section{Introducción} \label{sec:plan-intro}
En el este capitulo se describe el resultado obtenido tras la planificación del
proyecto \extranjerismo{software}.

La planificación toma dos vertientes. Primero se analiza la planificación 
temporal del proyecto. En esta fase se determina de manera general la cantidad
y el tipo de tareas a realizar. Una vez identificadas las tareas se puede
estimar el tiempo requerido para su realización. Agregando las estimaciones se
puede calcular el tiempo total estimado para realizar el proyecto completo.

Al usar la metodología Scrum, tras terminar cada \extranjerismo{sprint} se
evalúa si el desarrollo se está ciñendo a lo esperado y modificar las
estimaciones en consecuencia. Por lo que la planificación inicial puede
se puede desviar a lo largo del desarrollo.

Por otra parte se realiza un análisis de la viabilidad del proyecto. En todo
desarrollo se ven involucrados diferentes factores que pueden determinar la
viabilidad o no del proyecto. Los circunstancias analizadas son la viabilidad
económica y la viabilidad legal. Se busca conocer el coste económico estimado
en la realización del proyecto. Y también se quiere analizar el tipo de
licenciamiento \extranjerismo{software} que afecta al proyecto. Así como las
licencias que se van a asignar.



\section{Planificación temporal} \label{sec:plan-temporal}

Como se han empleado la metodología Scrum, la planificación se ha realizado
entorno a los \extranjerismo{sprints}. En la planificación inicial se
establecieron de forma general el número de \extranjerismo{sprints} y el
objetivo de cada uno de ellos.

La duración asignada a los \extranjerismo{sprints} se extendía desde una semana
a un mes. Para la gestión del proyecto se ha usado ZenHub y con él se pudo
asignar una estimación temporal a las tareas en forma de
\extranjerismo{Story points}. La puntuación va desde un punto, para las
tareas más simples y breves; hasta los cuarenta puntos, de las tareas más
complejas y que podían ocupar la mayor parte del tiempo del
\extranjerismo{sprint}.

La planificación inicial fue la siguiente:
\begin{itemize}
  \item \extranjerismo{Sprint} 1 Estudio e investigación
  \item \extranjerismo{Sprint} 2 Comunicarse con la placa usando TCP/IP 
  \item \extranjerismo{Sprint} 3 Implementar las funciones del 
    \extranjerismo{hardware}
  \item \extranjerismo{Sprint} 4 Desarrollar la aplicación web
  \item \extranjerismo{Sprint} 5 Finalizar la documentación de la memoria
  \item \extranjerismo{Sprint} 6 Finalizar la documentación de los apéndices
\end{itemize}

A continuación se muestra la planificación específica de cada
\extranjerismo{sprint} mostrando sus objetivos, las tareas que los componen,
la estimación de cada tarea y el resultado obtenido tras su finalización.

\clearpage

\subsection{Plan del \extranjerismo{Sprint} 1} \label{sec:plan-s1}
La estimación del \extranjerismo{sprint} se estableció en una semana de
duración. Con en este \extranjerismo{sprint} se realizaba la primera toma de
contacto con el proyecto. Las tareas a realizar giraban en torno al estudio de
la placa de desarrollo y sobre la elección del resto de herramientas, técnicas y
metodologías.

Acciones a realizar en este \extranjerismo{sprint}:
\begin{itemize}
  \item Estudio de las herramientas de trabajo.
  \item Investigación sobre la placa de desarrollo FRDM-K64F.
  \item Estudio sobre FreeRTOS y lwIP.
  \item Elección de las herramientas y \extranjerismo{software} a emplear.    
\end{itemize}

\imagen{sprint1}{Tareas planificadas para el Sprint 1}

Las tareas del \extranjerismo{sprint} se realizaron según lo previsto al plan.
El estudio sobre el IDE resultó la tarea más compleja de acuerdo a lo estimado,
pues hubo que aprender a usar el IDE y sus herramientas. Al terminar sin
imprevistos se pudo planear las tareas del siguiente \extranjerismo{sprint}.

\clearpage

\subsection{Plan del \extranjerismo{Sprint} 2} \label{sec:plan-s2}
La estimación del \extranjerismo{sprint} se estableció en una semana de
duración. Conociendo las herramientas con las que trabajar en este
\extranjerismo{sprint} se pretendía comenzar el desarrollo.

Acciones a realizar en este \extranjerismo{sprint}:
\begin{itemize}
  \item Creación del proyecto de trabajo.
  \item Configurar el \extranjerismo{hardware}.
  \item Realizar las primeras comunicaciones TCP/IP.  
  \item Realizar las operaciones remotas con la placa.
\end{itemize}

\imagen{sprint2}{Tareas planificadas para el Sprint 2}

El \extranjerismo{sprint} se completó de acuerdo a lo previsto. Aunque se 
estimó que la configuración del proyecto y el uso de lwIP no serían muy
difíciles, su ejecución resultó más compleja, sin embargo, se realizó a tiempo
para planificar y empezar el siguiente \extranjerismo{sprint}.

\clearpage

\subsection{Plan del \extranjerismo{Sprint} 3} \label{sec:plan-s3}
De nuevo, la estimación del \extranjerismo{sprint} se estableció en una semana
de duración. Como era posible transmitir datos a la placa, en este
\extranjerismo{sprint} se deseaba ampliar la funcionalidad de la placa K64F
con el uso de sus periféricos.

Acciones a realizar en este \extranjerismo{sprint}:
\begin{itemize}
  \item Configurar los LED RBG y realizar la programación para su manejo remoto.
  \item Configurar el LCD y realizar la programación para su manejo remoto. 
  \item Configurar los LED PWM y realizar la programación para su manejo remoto.
\end{itemize}

\imagen{sprint3}{Tareas planificadas para el Sprint 3}

Este fue el primer \extranjerismo{sprint} en desviarse de lo previsto. 
La configuración del bus I\textsuperscript{2}C y la adaptación de la librería
del LCD requirió mucho más tiempo de lo estimado. Al aumentar la duración
de este \extranjerismo{sprint} se tuvo que retrasar el inicio del siguiente.
Tras terminarlo, se pudieron planificar las tareas siguientes.

\clearpage

\subsection{Plan del \extranjerismo{Sprint} 4} \label{sec:plan-s4}
Con el cambio al desarrollo de la aplicación web y teniendo en cuenta la
posibilidad de nuevos retrasos como en el \extranjerismo{sprint} anterior,
en este se decidió aumentar la planificación temporal a dos semanas.

Acciones a realizar en este \extranjerismo{sprint}:
\begin{itemize}
  \item Analizar el intercambio de datos necesario entre app y placa.
  \item Diseñar las interfaces de la aplicación. 
  \item Implementar las funciones de comunicación.
  \item Desarrollar la interfaz web.
\end{itemize}

\imagen{sprint4}{Tareas planificadas para el Sprint 4}

El desarrollo de la aplicación presentó varios desafíos, el aprendizaje y 
programación con CSS fueron algunos de ellos. Pese a todo la duración no 
se desvió demasiado de lo previsto.

\clearpage

\subsection{Plan del \extranjerismo{Sprint} 5} \label{sec:plan-s5}
Terminados el desarrollo del \extranjerismo{sofware} de la placa y 
el desarrollo de la aplicación web faltaba completar la documentación del 
proyecto. Inicialmente se planificó que el \extranjerismo{sprint}
tendría una duración de dos semanas.

Acciones a realizar en este \extranjerismo{sprint}:
\begin{itemize}
  \item Completar los distintos capítulos de la memoria.
\end{itemize}

\imagen{sprint5}{Tareas planificadas para el Sprint 5}

El tiempo dedicado a preparase para usar \LaTeX{} fue erróneamente subestimado.
Al tener que dedicar varios días más de lo previsto en este punto, el último
\extranjerismo{sprint} tuvo que empezar más tarde.

\clearpage

\subsection{Plan del \extranjerismo{Sprint} 6} \label{sec:plan-s6}
Completada la memoria se completaría la información proporcionada con una
serie de anexos. La duración se estimó en una semana debido a los plazos
temporales restantes.

Acciones a realizar en este \extranjerismo{sprint}:
\begin{itemize}
  \item Completar los distintos apéndices.
\end{itemize}

\imagen{sprint6}{Tareas planificadas para el Sprint 6}

Completada la documentación se pudo proceder a la conclusión satisfactoria
del proyecto.

\clearpage

\subsection{Planificación temporal completa} \label{sec:plan-total}
En las siguientes páginas se pueden comparar la planificación inicial
y la final.

\imagen{tareas1}{Sprints planificados inicialmente}

\imagen{tareas2}{Sprints planificados finalmente}

\figuraApaisadaSinMarco{0.7}{gantt1}{Diagrama de Gantt de la planificación
inicial}{fig:gantt1}{}

\figuraApaisadaSinMarco{0.8}{gantt2}{Diagrama de Gantt de la planificación
final}{fig:gantt2}{}

\clearpage



\section{Estudio de viabilidad}


\subsection{Viabilidad económica}
Para determinar la viabilidad económica del proyecto, en este apartado se
computan los gastos previstos durante el desarrollo y los posibles beneficios
si los hubiera. De haberse realizado en una empresa, los costes se calcularían
como se muestra a continuación.

\subsubsection{Coste de personal}
El desarrollo se ha realizado por una sola persona en un tiempo aproximado de
90 días, así que, todos los costes de personal son relativos a esa única
persona. El salario mensual bruto asignado ha sido de \EUR{2200}.

De manera simplificada, sin calcular  el salario neto se puede calcular de la siguiente forma:
\begin{equation} \label{eq:salario}
  salario\ bruto - IRPF - SS = salario\ neto
\end{equation}

Por tanto, para un mes dado el salario neto a percibir sería el siguiente:
\begin{equation} \label{eq:salario}
  2200\text{\euro} - 29\% \footnote{Usados los tramos más altos de las
  escalas autonómica \cite{webpage:cyl} y estatal \cite{webpage:at} aplicables,
  el cálculo está simplificado al no tener en cuenta la progresividad de
  la escala de tramos.}
  - 6,35\% = 1422,30\text{\euro}
\end{equation}

\subsubsection{Costes pertenecientes a la SS}
Sobre el salarios se producen ciertas retenciones y pagos a la Seguridad Social,
algunos conceptos corren a cargo de la empresa contratante y otros
por parte del trabajador. A continuación de calculan sus importes.

Para el cálculo de la retenciones se toma como referencia la tablas de bases
de cotización ofrecida por la propia Seguridad Social \cite{webpage:ss}. En
concreto los datos del primer grupo de cotización: ``Ingenieros y Licenciados. 
Personal de alta dirección no incluido en el artículo 1.3.c)
del Estatuto de los Trabajadores''. Este grupo tiene unas bases mínimas de
\EUR{1466,40}/mes. Y unas máximas de \EUR{4070,10}/mes.

\tablaSmallSinColores{Costes pertenecientes a la SS}{l l l}{costes-ss}
{\multicolumn{1}{l}{Concepto} & Empresa & Trabajador\\}
{
  Contingencias comunes & 23,60\% & 4,70\%\\
  Desempleo             &  5,50\% & 1,55\%\\
  FOGASA                &  0,20\% & 0,00\%\\
  Formación             &  0,60\% & 0,10\%\\
  \textbf{Total}        & \textbf{29,9\%} & \textbf{6,35\%}\\
}

Por consiguiente, la empresa tendría que añadir a los costes el 23,9\% del
salario del trabajador y el trabajador sufriría una retención del 6,35\%.

\subsubsection{Coste total de personal}
El coste total se calcula con la siguiente fórmula:
\begin{equation} \label{eq:salario}
  (salario\ mensual + retenciones\ ss) \times n^{o}\ meses = coste\ total
\end{equation}

De modo que el coste total se obtiene así:
\begin{equation} 
  (2200\text{\euro} + 657,8\text{\euro}) \times 3 = 8573,4\text{\euro}
\end{equation}

La tabla siguiente tabla recoge diferentes costes asociados al personal y 
su coste total.

\tablaSmallSinColores{Coste total de personal}{l l}{personal}
{\multicolumn{1}{l}{Concepto} & Coste\\}
{
  Salarios          & \EUR{2200}\\
  Seguridad Social  & \EUR{657,8}\\
  Meses             & 3 meses\\
  \textbf{Total}    & \textbf{\EUR{8573,4}}\\
}

\subsubsection{Coste del \extranjerismo{hardware}}
Para realizar el desarrollo se precisa contar son varios dispositivos. Aunque
en la Ley del Impuesto sobre Sociedades \cite{webpage:is} se indica un máximo
de 8 años para amortizar los equipos para procesos de información, se considera
la que renovación de \extranjerismo{hardware} se produce en un periodo más corto
de 4 años. Así pues, la amortización se calcula en función de esos 4 años.

El coste mensual de un dispositivo se calcula así:
\begin{equation} \label{eq:coste-hw}
  coste\ del\ dispositivo\ /\ periodo\ de\ amortización\ (en\ meses)
\end{equation}

Por lo tanto, el coste de un dispositivo es el siguiente:
\begin{equation} \label{eq:coste-amor-hw}
  coste\ mensual\ amortizado\ \times n^{o}\ meses\
\end{equation}

En el desarrollo solo se va a necesitar una estación de trabajo, así pues, el
coste total del \extranjerismo{hardware} es el mostrado en la tabla a
continuación.

\tablaSmallSinColores{Coste del \extranjerismo{hardware}}{l l l}{coste-hw}
{\multicolumn{1}{l}{\extranjerismo{Hardware}} & Coste & Coste amortizado\\}
{
  Estación de trabajo & \EUR{1250} & \EUR{104,16}\\
  \textbf{Total}      & \EUR{1250} & \textbf{\EUR{104,16}}\\
}

\subsubsection{Coste del \extranjerismo{software}}
El desarrollo necesita de varios programas y aplicaciones. De igual manera que
el \extranjerismo{hardware}, la Ley del Impuesto sobre Sociedades
\cite{webpage:is} indica el máximo de años para realizar la amortización.
En los sistemas y programas informáticos es de 6 años. Pero coincidiendo con el
periodo de renovación del \extranjerismo{hardware}, la amortización se calcula
en un periodo menor de 4 años.

Las fórmulas para calcular las amortizaciones son equivalentes a las usadas
en el \extranjerismo{harware} \ref{eq:coste-hw} y \ref{eq:coste-amor-hw}.

Otra parte de \extranjerismo{software} se emplea bajo suscripción, en los
cálculos solo se computa el coste de los meses que se usó dicha suscripción.

Hay que tener en cuenta, que gran parte del \extranjerismo{software} utilizado
se publica bajo licencias que permiten su uso sin coste y que gracias a esto se
reducen significativamente los costes del \extranjerismo{software}.
En consecuencia solo se calculan los costes del \extranjerismo{software} que no
es gratuito.

\tablaSmallSinColores{Coste del \extranjerismo{software}}{l l l}{coste-sw}
{\multicolumn{1}{l}
{\extranjerismo{Software}}              & Coste        & Coste amortizado\\}
{
  Windows 10 Pro\cite{webpage:win10pro} & \EUR{259}    & \EUR{21,58}\\
  Office 365\cite{webpage:office365}    & \EUR{126}    & \EUR{31,5}\\
  Adobe CC\cite{webpage:adobecc}        & \EUR{359,88} & \EUR{89,97}\\
  \textbf{Total}                        & \EUR{744,88} & \textbf{\EUR{143,05}}\\
}

\subsubsection{Coste del sistema empotrado}
El sistema empotrado se ha realizado con la ayuda de diversos componentes que
también tienen unos costes de adquisición asociados.

\tablaSmallSinColores{Coste del SE}{l l}{coste-se}
{\multicolumn{1}{l}
{Componente}                                     & Coste        \\}
{ 
  Placa FRDM-K64F\cite{webpage:k64f}             & \EUR{32,52} \\
  Placa Arduino Basic I/O\cite{webpage:basic-io} & \EUR{19,00} \\
  Pantalla LCD\cite{webpage:lcd}                 & \EUR{8,91}  \\
  Placas de pruebas\cite{webpage:placas}         & \EUR{12,24} \\
  Cables puente\cite{webpage:cables}             & \EUR{4,98}  \\
  Cable de par trenzado\cite{webpage:rj45}       & \EUR{4,88}  \\
  \textbf{Total}                                 & \textbf{\EUR{82,53}} \\
}

\subsubsection{Coste total del proyecto}
Sumando todos los costes anteriores se puede calcular el coste total de todo
el proyecto.

\tablaSmallSinColores{Coste total del proyecto}{l l}{coste-total}
{\multicolumn{1}{l}
{Tipo de coste}            & Coste        \\}
{ 
  Personal                 & \EUR{8573,4} \\
  \extranjerismo{Hardware} & \EUR{104,16} \\
  \extranjerismo{Software} & \EUR{143,05} \\
  Componentes del SE       & \EUR{82,53}  \\
  \textbf{Total}           & \textbf{\EUR{8903,14}} \\
}

\subsubsection{Beneficios del proyecto}
El código fuente del sistema empotrado (SE) y de la aplicación web se encuentran
disponibles abiertamente. Además, el SE tampoco tiene definida una función
claramente comercial. Por ello, no se considera que el proyecto
pueda tener un beneficio económico directo.

Esto tampoco quiere decir que los gastos se hagan a fondo perdido. Más bien,
se pueden considerar a los gastos como una inversión en investigación, formación
y adquisición de competencias que permitirán en el futuro crear nuevos SE que
rentabilicen todos los costes.


\subsection{Viabilidad legal}
Al desarrollar un \extranjerismo{software} hay que tener en consideración
las implicaciones legales que se presentan al usar \extranjerismo{software}
de terceros.

Las licencias sirven como instrumento para establecer los términos y condiciones
en los que se pueden utilizar el \extranjerismo{software} licenciado. Para
poder definir que licencias usar en el proyecto es necesario conocer las
licencias utilizadas y la restricciones que pueden imponer.

\subsubsection{Licencias utilizadas en el desarrollo del SE}
Para desarrollar el \extranjerismo{software} de la placa se ha utilizado
de código fuente de terceros que se describe a continuación.
\begin{itemize}
  \item El código generado por MCUXpresso se proporciona bajo la licencia
  ``The 3-Clause BSD License'' (BSD-3), también conococida como la
  ``New BSD License''o ``Modified BSD License''.
  \item Otra parte del código generado por MCUXpress se licencia con 
  ``The Clear BSD License''. La licencia es similar a BSD-3, pero indica
  expresamente que no se conceden derechos sobre patentes.
  \item El código perteneciente a FreeRTOS es propiedad de Amazon.com, Inc.
  Se permite el uso, copia, modificación, publicación, distribución, volver a
  licenciar y comercializar, manteniendo siempre el aviso de derechos de autor.
  \item El código de lwIP se entrega con la licencia BSD-3.
  \item Parte del código relacionado con el procesador de la placa pertenece a
  ARM Limited. El código se proporciona con la licencia Apache 2.0.
\end{itemize}

Así pues, se encuentran licencias como BSD-3 o Apache 2.0 que permiten el uso,
copia, modificación y redistribución del código.

\subsubsection{Licencia para el SE}
Teniendo en cuanta que el código fuente del \extranjerismo{software} del SE
se pretende que sea abierto, se va a utilizar una licencia en línea con las
mostradas anteriormente.

En concreto se va a emplear la licencia Apache 2.0. Sus características 
principales se pueden ver resumidas en la siguiente tabla.

\tablaSmallSinColores{Licencia Apache 2.0}{l l l l}{apache}
{\multicolumn{1}{l}
{Permisos}        & Condiciones                & Limitaciones    \\}
{ 
  Uso comercial   & Aviso de licencia          & Uso de marcas registradas \\
  Modificación    & y derechos de autor        & Responsabilidad           \\
  Distribución    & Declaración de los cambios & Garantía                  \\ 
  Uso en patentes \\
  Uso privado     \\
}

\subsubsection{Licencia para la aplicación web}
En el caso de la aplicación web no se ha requerido de bibliotecas o librerías
de terceros como en el caso del SE. Del mismo modo que en el SE, la aplicación
web se licencia con Apache 2.0.

\subsubsection{Utilización de la licencia Apache}
Para poner en conocimiento el uso de la licencia Apache
\cite{webpage:apache2-apply} se ubica en el directorio raíz del proyecto un
archivo de texto con el nombre ``LICENSE'' que incluye los términos de la
licencia. El archivo de texto se puede obtener desde el sitio web de Apache
\cite{webpage:apache2-license}.

En caso de ser necesario, se incluye un fichero llamado ``NOTICE'' con
información adicional en el mismo directorio de la licencia.

Por último, el texto a continuación es incluido en todos los ficheros 
fuente. Se tiene que añadir como comentario en el comienzo del fichero,
sustituyendo el texto entre corchetes por el nombre del autor y por la fecha
que correspondan.

\begin{quotation}
  Copyright [yyyy] [name of copyright owner] \bigskip

  Licensed under the Apache License, Version 2.0 (the "License");
  you may not use this file except in compliance with the License.
  You may obtain a copy of the License at \bigskip
  
  \quad http://www.apache.org/licenses/LICENSE-2.0 \bigskip

  Unless required by applicable law or agreed to in writing, software
  distributed under the License is distributed on an ``AS IS'' BASIS,
  WITHOUT WARRANTIES OR CONDITIONS OF ANY KIND, either express or implied.
  See the License for the specific language governing permissions and
  limitations under the License.
\end{quotation}

\subsubsection{Licencia para la documentación}
Para el licenciamiento de la documentación se opta por otro tipo de licencia
más adecuada al tipo de obra. La licencia escogida es
Creative Commons Atribución-NoComercial-CompartirIgual 4.0 Internacional
(CC BY-NC-SA 4.0) \cite{webpage:cc}.

Esta licencia permite que la obra sea compartida libremente. Puede ser copiada
y redistribuida en cualquier medio o formato. También se puede reeditar,
transformar y crear obras derivadas a partir de la obra original. Como condición
se requiere la atribución de la autoría original y que nuevas publicaciones
se realicen de forma gratuita y con el mismo licenciamiento que la actual.

\tablaSmallSinColores{Licencia CC BY-NC-SA 4.0}{l l l l}{ccbyncsa4}
{\multicolumn{1}{l}
{Permisos}      & Condiciones                & Limitaciones    \\}
{ 
  Distribución  & Crédito al autor           & Responsabilidad           \\  
  Modificación  & Aviso de licencia          & Uso de patentes           \\
  Uso privado   & y derechos de autor        & Uso de marcas registradas \\
                & Declaración de los cambios & Garantía                  \\ 
                & No comercial               &                           \\ 
                & Mismo licenciamiento       &                           \\ 
}

El uso de la licencia se indica con una imagen y un breve texto al final de cada
documento.
