\documentclass[a4paper,12pt,twoside]{memoir}

% Castellano
\usepackage[spanish,es-tabla]{babel}
\selectlanguage{spanish}
\usepackage[utf8]{inputenc}
\usepackage[T1]{fontenc}
\usepackage{lmodern} % Scalable font
\usepackage{microtype}
\usepackage{placeins}
%\usepackage{url} % Mejora el tratamiento de URL, p. ej. en saltos de línea.
\usepackage{float}
\usepackage{subfig}

 

\RequirePackage{booktabs}
\RequirePackage[table]{xcolor}
\RequirePackage{xtab}
\RequirePackage{multirow}

% Links
\PassOptionsToPackage{hyphens}{url}
\usepackage[colorlinks]{hyperref}
\hypersetup{
	allcolors = {red}
}

% Ecuaciones
\usepackage{amsmath}

% Rutas de fichero / paquete
\newcommand{\ruta}[1]{{\sffamily #1}}

% Párrafos
\nonzeroparskip

% Imágenes
\usepackage{graphicx}
\newcommand{\imagen}[2]{
	\begin{figure}[!h]
		\centering
		\includegraphics[width=0.9\textwidth]{#1}
		\caption{#2}\label{fig:#1}
	\end{figure}
	\FloatBarrier
}

\newcommand{\imagenflotante}[2]{
	\begin{figure}%[!h]
		\centering
		\includegraphics[width=0.9\textwidth]{#1}
		\caption{#2}\label{fig:#1}
	\end{figure}
}

% El comando \figura nos permite insertar figuras cómodamente, y utilizando
% siempre el mismo formato. Los parámetros son:
% 1 -> Porcentaje del ancho de página que ocupará la figura (de 0 a 1)
% 2 --> Fichero de la imagen
% 3 --> Texto a pie de imagen
% 4 --> Etiqueta (label) para referencias
% 5 --> Opciones que queramos pasar a \includegraphics
% 6 --> Opciones de posicionamiento a pasar a \begin{figure}
\newcommand{\figuraConPosicion}[6]{%
  \setlength{\anchoFloat}{#1\textwidth}%
  \addtolength{\anchoFloat}{-4\fboxsep}%
  \setlength{\anchoFigura}{\anchoFloat}%
  \begin{figure}[#6]
    \begin{center}%
      \Ovalbox{%
        \begin{minipage}{\anchoFloat}%
          \begin{center}%
            \includegraphics[width=\anchoFigura,#5]{#2}%
            \caption{#3}%
            \label{#4}%
          \end{center}%
        \end{minipage}
      }%
    \end{center}%
  \end{figure}%
}

%
% Comando para incluir imágenes en formato apaisado (sin marco).
\newcommand{\figuraApaisadaSinMarco}[5]{%
  \begin{figure}%
    \begin{center}%
    \includegraphics[angle=90,height=#1\textheight,#5]{#2}%
    \caption{#3}%
    \label{#4}%
    \end{center}%
  \end{figure}%
}
% Para las tablas
\newcommand{\otoprule}{\midrule [\heavyrulewidth]}
%
% Nuevo comando para tablas pequeñas (menos de una página).
\newcommand{\tablaSmall}[5]{%
 \begin{table}[h]
  \begin{center}
   \rowcolors {2}{gray!35}{}
   \begin{tabular}{#2}
    \toprule
    #4
    \otoprule
    #5
    \bottomrule
   \end{tabular}
   \caption{#1}
   \label{tabla:#3}
  \end{center}
 \end{table}
}

%
% Nuevo comando para tablas pequeñas (menos de una página).
\newcommand{\tablaSmallSinColores}[5]{%
 \begin{table}[H]
  \begin{center}
   \begin{tabular}{#2}
    \toprule
    #4
    \otoprule
    #5
    \bottomrule
   \end{tabular}
   \caption{#1}
   \label{tabla:#3}
  \end{center}
 \end{table}
}

\newcommand{\tablaApaisadaSmall}[5]{%
\begin{landscape}
  \begin{table}
   \begin{center}
    \rowcolors {2}{gray!35}{}
    \begin{tabular}{#2}
     \toprule
     #4
     \otoprule
     #5
     \bottomrule
    \end{tabular}
    \caption{#1}
    \label{tabla:#3}
   \end{center}
  \end{table}
\end{landscape}
}

%
% Nuevo comando para tablas grandes con cabecera y filas alternas coloreadas en gris.
\newcommand{\tabla}[6]{%
  \begin{center}
    \tablefirsthead{
      \toprule
      #5
      \otoprule
    }
    \tablehead{
      \multicolumn{#3}{l}{\small\sl continúa desde la página anterior}\\
      \toprule
      #5
      \otoprule
    }
    \tabletail{
      \hline
      \multicolumn{#3}{r}{\small\sl continúa en la página siguiente}\\
    }
    \tablelasttail{
      \hline
    }
    \bottomcaption{#1}
    \rowcolors {2}{gray!35}{}
    \begin{xtabular}{#2}
      #6
      \bottomrule
    \end{xtabular}
    \label{tabla:#4}
  \end{center}
}

%
% Nuevo comando para tablas grandes con cabecera.
\newcommand{\tablaSinColores}[6]{%
  \begin{center}
    \tablefirsthead{
      \toprule
      #5
      \otoprule
    }
    \tablehead{
      \multicolumn{#3}{l}{\small\sl continúa desde la página anterior}\\
      \toprule
      #5
      \otoprule
    }
    \tabletail{
      \hline
      \multicolumn{#3}{r}{\small\sl continúa en la página siguiente}\\
    }
    \tablelasttail{
      \hline
    }
    \bottomcaption{#1}
    \begin{xtabular}{#2}
      #6
      \bottomrule
    \end{xtabular}
    \label{tabla:#4}
  \end{center}
}

%
% Nuevo comando para tablas grandes sin cabecera.
\newcommand{\tablaSinCabecera}[5]{%
  \begin{center}
    \tablefirsthead{
      \toprule
    }
    \tablehead{
      \multicolumn{#3}{l}{\small\sl continúa desde la página anterior}\\
      \hline
    }
    \tabletail{
      \hline
      \multicolumn{#3}{r}{\small\sl continúa en la página siguiente}\\
    }
    \tablelasttail{
      \hline
    }
    \bottomcaption{#1}
  \begin{xtabular}{#2}
    #5
   \bottomrule
  \end{xtabular}
  \label{tabla:#4}
  \end{center}
}



\definecolor{cgoLight}{HTML}{EEEEEE}
\definecolor{cgoExtralight}{HTML}{FFFFFF}

%
% Nuevo comando para tablas grandes sin cabecera.
\newcommand{\tablaSinCabeceraConBandas}[5]{%
  \begin{center}
    \tablefirsthead{
      \toprule
    }
    \tablehead{
      \multicolumn{#3}{l}{\small\sl continúa desde la página anterior}\\
      \hline
    }
    \tabletail{
      \hline
      \multicolumn{#3}{r}{\small\sl continúa en la página siguiente}\\
    }
    \tablelasttail{
      \hline
    }
    \bottomcaption{#1}
    \rowcolors[]{1}{cgoExtralight}{cgoLight}

  \begin{xtabular}{#2}
    #5
   \bottomrule
  \end{xtabular}
  \label{tabla:#4}
  \end{center}
}

\graphicspath{ {./img/} }

% Capítulos
\chapterstyle{bianchi}
\newcommand{\capitulo}[2]{
	\setcounter{chapter}{#1}
	\setcounter{section}{0}
	\chapter*{#2}
	\addcontentsline{toc}{chapter}{#2}
	\markboth{#2}{#2}
}

% Apéndices
\renewcommand{\appendixname}{Apéndice}
\renewcommand*\cftappendixname{\appendixname}

\newcommand{\apendice}[1]{
	%\renewcommand{\thechapter}{A}
	\chapter{#1}
}

\renewcommand*\cftappendixname{\appendixname\ }

% Formato de portada
\makeatletter
\usepackage{xcolor}
\newcommand{\tutor}[1]{\def\@tutor{#1}}
\newcommand{\course}[1]{\def\@course{#1}}
\definecolor{cpardoBox}{HTML}{E6E6FF}
\def\maketitle{
  \null
  \thispagestyle{empty}
  % Cabecera ----------------
\noindent\includegraphics[width=\textwidth]{cabecera}\vspace{1cm}%
  \vfill
  % Título proyecto y escudo informática ----------------
  \colorbox{cpardoBox}{%
    \begin{minipage}{.8\textwidth}
      \vspace{.5cm}\Large
      \begin{center}
      \textbf{TFM del Máster Universitario en Ingeniería Informática}\vspace{.6cm}\\
      \textbf{\LARGE\@title{}}
      \end{center}
      \vspace{.2cm}
    \end{minipage}

  }%
  \hfill\begin{minipage}{.20\textwidth}
    \includegraphics[width=\textwidth]{escudoInfor}
  \end{minipage}
  \vfill
  % Datos de alumno, curso y tutores ------------------
  \begin{center}%
  {%
    \noindent\LARGE
    Presentado por \@author{}\\ 
    en Universidad de Burgos --- \@date{}\\
    Tutor: \@tutor{}\\
  }%
  \end{center}%
  \null
  \cleardoublepage
  }
\makeatother

\newcommand{\nombre}{Enrique del Olmo Domínguez} %%% cambio de comando
     
% Comando para formatear palabras procedentes de otros
% lenguajes distintos del castellano.
% Estas palabras se pueden escribir en letra cursiva 
\newcommand{\extranjerismo}[1]{\textit{#1}}
% o con comillas si no se dispone de cursiva.
%\newcommand{\extranjerismo}[1]{"{#1}"}

% Comando para formatear los títulos de otras obras de creación.
\newcommand{\titulo}[1]{\textit{#1}}

% Datos de portada
\title{Comunicación TCP/IP con sistemas empotrados}
\author{\nombre}
\tutor{Alejandro Merino Gómez y Daniel Sarabia Ortiz}
\date{\today}

\begin{document}

\maketitle

\newpage\null\thispagestyle{empty}

%%%%%%%%%%%%%%%%%%%%%%%%%%%%%%%%%%%%%%%%%%%%%%%%%%%%%%%%%%%%%%%%%%%%%%%%%%%%%%%%%%%%%%%%
\thispagestyle{empty}


\noindent\includegraphics[width=\textwidth]{cabecera}\vspace{1cm}

\noindent D. Alejandro Merino Gómez y D. Daniel Sarabia Ortiz, profesores del departamento de Ingeniería Electromecánica (Área de Ingeniería de Sistemas y Automática).

\noindent Exponen:

\noindent Que el alumno D. \nombre, con DNI 71309191Z, ha realizado el Trabajo final de Grado en Ingeniería Informática titulado Comunicación con Sistemas Embebidos. 

\noindent Y que dicho trabajo ha sido realizado por el alumno bajo la dirección de los que suscriben, en virtud de lo cual se autoriza su presentación y defensa.

\begin{center} %\large
En Burgos, {\large \today}
\end{center}

\vfill\vfill\vfill

% Author and supervisor
\begin{minipage}{0.45\textwidth}
\begin{flushleft} %\large
Vº. Bº. del Tutor:\\[2cm]
D. Alejandro Merino Gómez
\end{flushleft}
\end{minipage}
\hfill
\begin{minipage}{0.45\textwidth}
\begin{flushleft} %\large
Vº. Bº. del co-tutor:\\[2cm]
D. Daniel Sarabia Ortiz
\end{flushleft}
\end{minipage}
\hfill

\vfill

% para casos con solo un tutor comentar lo anterior
% y descomentar lo siguiente
%Vº. Bº. del Tutor:\\[2cm]
%D. nombre tutor


\newpage\null\thispagestyle{empty}\newpage




\frontmatter

% Abstract en castellano
\renewcommand*\abstractname{Resumen}
\begin{abstract}
Los sistemas embebidos (SE) son pequeños controladores capaces de realizar unas tareas determinadas programadas con anterioridad. Este tipo de sistemas son muy utilizados en la actualidad ya que, su hardware, bajo consumo, facilidad de uso y su escalabilidad los hacen propicios para solucionar un multitud de tareas. Estas tareas pueden ser desde la inclinación de una cama de hospital a la gestión de unas cintas transportadoras de elementos en una industria.
En este proyecto podremos ver como se pueden programar sistemas embebidos para comunicarse entre sí mediante Internet. También se mostrará una simulación de para qué se podrían utilizar este tipo de sistemas en un entorno real. Para ello dispondremos de un laboratorio de pruebas con varias placas FRDM-K64F, dos servomotores, una pantalla LCD y un `shield' de expansión.
\end{abstract}

\renewcommand*\abstractname{Descriptores}
\begin{abstract}
Sistemas embebidos, Sistemas empotrados, conexión ethernet, comunicación entre sistemas, Internet de las cosas (IoT), NXP, UART, I2C, ADC, FRDM K64F.
\end{abstract}

\clearpage

% Abstract en inglés
\renewcommand*\abstractname{Abstract}
\begin{abstract}
Embedded systems (ES) are small controllers capable of performing specific tasks programmed in advance. This type of systems are widely used nowadays because their hardware, low power consumption, ease of use and scalability make them suitable to solve a lot of tasks. These tasks can be from the inclination of a hospital bed to the management of conveyor belts of elements in an industry.
In this project we will see how embedded systems can be programmed to communicate with each other over the Internet. We will also see a simulation of what these types of systems could be used for in a real environment. For this we will have a test laboratory with several FRDM-K64F boards, two servomotors, an LCD screen and an expansion shield.
\end{abstract}

\renewcommand*\abstractname{Keywords}
\begin{abstract}
Embedded systems, ethernet connection, inter-system communication, Internet of Things (IoT), NXP, UART, I2C, ADC, FRDM K64F.
\end{abstract}

\clearpage

% Índices
\tableofcontents

\clearpage

\listoffigures

\clearpage

\listoftables
\clearpage

\mainmatter
\capitulo{1}{Introducción}

Los sistemas embebidos o empotrados (SE) están muy presentes en nuestra vida cotidiana \cite{UnicanSE}. Algunos ejemplos de sistemas empotrados que podemos encontrar en nuestro día a día serían los electrodomésticos, relojes, coches, semáforos, entre otros. Todos estos aparatos, junto con robots o máquinas industriales, componen un campo importante para nuestra sociedad, ya que estos sistemas facilitan enormemente tareas pesadas o repetitivas en la vida de las personas.

Podemos referirnos a estos sistemas como un microcontrolador. Es importante conocer las diferencias entre un microcontrolador y un microcomputador \cite{ALMCMP}. Como diferencias a nivel técnico tenemos:
 
\begin{itemize}
\item En lo referido al software, encontramos que los microcontroladores tienen como objetivo la realización de pequeñas tareas programadas por un desarrollador para un fin concreto. Es por ello que cuentan con algunas características a nivel general de todos los microcontroladores como puede ser, bajo consumo, tamaño reducido y bajo coste. En cambio, un microcomputador se suele utilizar para entornos complejos y tareas extensas que a su vez requieren de otras tareas en cascada.
\item A nivel de hardware, encontramos que un microcontrolador engloba tanto la unidad de procesamiento como una pequeña unidad de memoria (ROM, RAM, etc), además de algunos puertos para periféricos, un temporizador, etc. Podemos pensar en un microcontrolador como una minicomputadora.
\end{itemize}

En cuanto a los periféricos, se les pueden añadir sensores dotándoles de nuevas funcionalidades como por ejemplo, medición de humedad y calor, envío y recibo de ultrasonidos y comunicaciones bluetooth o infrarroja y un largo etc.  

Para que todos estos dispositivos puedan funcionar adecuadamente, se necesitan o al menos se prefiere tener a estos sistemas conectados entre sí. Mediante la transferencia de datos se pueden implementar aún más procesos, siempre con el objetivo de ser más eficiente a la hora de solucionar un problema.

Ahora que ya hemos visto su importancia y uso en la actualidad, veamos cómo funcionan los sistemas empotrados en tiempo real. En la mayoría de los casos en los que se utilizan sistemas embebidos, se requiere que realicen las operaciones rápidamente. Una instrucción debe ser ejecutada de manera inmediata o con un retardo mínimo. Para conseguirlo se suelen utilizar Sistemas Operativos en Tiempo Real (RTOS) para la gestión del tiempo de ejecución de cada tarea.

 Los SE son parte central de este nuevo mundo interconectado, en el cual todos nuestros aparatos electrónicos se comunican entre ellos mediante un hardware específico y un software que cumple con tareas en tiempo real. Esta tecnología cada vez está más presente en nuestros hogares y sin duda es un gran avance hacia el bienestar de las personas.


\section{Descripción del trabajo realizado}\label{sec:Descripcion}

La idea principal de este proyecto es disponer de varios sistemas embebidos conectados en red, que se comuniquen mediante cable ethernet. Esta conexión será de tipo punto a punto. Dispondremos de tres placas FRDM-K64F. A una de ellas se le añadirá además un \extranjerismo{shield} de expansión que la dotará de más periféricos, de las cuales usaremos los 4 botones, las 4 luces led de cara a la interacción con el usuario, además de los dos potenciómetros y el sensor de temperatura. 
En el caso de las otras dos placas, disponen de una pantalla LCD y además están conectadas a una placa controladora de dos motores EMG30. 

La Figura \ref{esqCon} muestra el diagrama de conexiones para ver cómo está construida, la planta piloto.

%\imagen{esquemaLaboratorio}{Esquema de conexiones del Laboratorio}\label{esqCon}
\begin{figure}[!h]
	\centering
	\includegraphics[width=0.9\textwidth]{esquemaLaboratorio}
	\caption{Esquema de conexiones del laboratorio.}\label{esqCon}
\end{figure}

Como podemos observar, la planta piloto consta de una red local cableada de tipo ethernet. La red está compuesta por tres sistemas empotrados conectados entre sí mediante un switch a través de cable ethernet, siguiendo una topología de estrella. La comunicación se realiza terminal a terminal, pasando por un switch, que reenvía los paquetes enviados de forma que lleguen correctamente a su destinatario.
La placa de color verde que vemos en la imagen anterior es el shield de expansión que está colocado encima de la placa maestro. El shield contiene los dos potenciómetros, 4 botones, 4 leds y el sensor de temperatura con los que se interactuará con el resto de los sistemas embebidos.
Por último, la Figura \ref{PlantaPiloto} muestra la planta piloto.

%\imagen{proyecto}{Planta piloto del proyecto.}\label{PlantaPiloto}
\begin{figure}[!h]
	\centering
	\includegraphics[width=0.9\textwidth]{proyecto}
	\caption{Planta piloto del proyecto.}\label{PlantaPiloto}
\end{figure}

El funcionamiento consta de los siguientes casos de uso:
\begin{itemize}
\item Mediante el potenciómetro 1 fijaremos la velocidad y sentido del giro del motor A. Una vez regulado enviaremos la petición pulsando el botón 1.
\item El potenciómetro 2 y el botón 2 se utilizan de la misma manera para fijar la velocidad del motor B.
\item El botón 3 nos sirve como parada de emergencia para los dos motores.
\item Al pulsar el botón 4 la placa maestro reporta la temperatura captada por el sensor de temperatura y se muestra por pantalla.
\item El botón 5 y 6 realizan una petición a la placa controladora de los motores para saber a qué velocidad giran el motor A y el motor B respectivamente.
\end{itemize}

\section{Estructura de la memoria}\label{sec:Estructura}

La memoria de este proyecto está estructurada de la siguiente manera.

\begin{description}
	\item[Introducción.] Se explican los temas principales en los que se basa tanto la idea como el procedimiento y funcionalidad final del proyecto.

	\item[Objetivos del proyecto.] Se explican los objetivos del proyecto tanto generales, como técnicos y personales, que se esperan cumplir durante la realización del proyecto.

	\item[Conceptos teóricos.] Se tratan los conceptos teóricos más detalladamente. Conceptos como el protocolo TCP/IP, las conexiones en red y comunicaciones con otros periféricos.

	\item[Técnicas y herramientas.] En este apartado se verán las herramientas utilizadas para llevar a cabo este proyecto, así como las técnicas para la correcta organización y gestión del trabajo.

	\item[Aspectos Relevantes del desarrollo.] En este capítulo se mostrarán algunos conceptos e hitos importantes durante el desarrollo del trabajo.

	\item[Trabajos relacionados.] Se mostrarán algunos ejemplos de uso real en la actualidad, que utilizan sistemas empotrados para su desarrollo

	\item[Conclusiones y líneas de trabajo futuras.] Se mencionarán algunas propuestas sobre hacia donde podría evolucionar el trabajo expuesto en esta memoria y a que conclusiones se ha llegado tras haber completado el objetivo del trabajo.
\end{description}

\section{Anexos}\label{sec:anexos}
Los anexos consisten en:
\begin{description}
	\item[Plan del proyecto software.] Se muestra la planificación temporal y la viabilidad del proyecto dividida en dos partes, legal y económica.

	\item[Especificación de requisitos del software.] Se expone un catálogo de requisitos para la utilización de este sistema en un entorno real. Este catálogo viene con la definición de cada una de sus partes.

	\item[Especificación de diseño.] Exposición de la fase de diseño, el plan procedimental y el diseño arquitectónico de las placas y sensores que componen el sistema.

	\item[Manual del programador.] Explicación de aquellos conocimientos e ideas que un programador debería conocer para poder entender y seguir desarrollando el código fuente.

	\item[Manual de usuario.] Contiene una explicación sobre el debido uso y pasos a seguir para que un usuario pueda utilizar adecuadamente el software.
\end{description}

\section{Contenido adjunto}\label{sec:Contenido adjunto}
\begin{enumerate}
\item Software para las placas `Shield' y `Placa motor A' y `Placa motor B'.
\item Video del funcionamiento de la planta piloto compuesta por dos motores y su placa controladora, una pantalla LCD, tres placas k64f y una placa de expansión.
\item Repositorio de GitHub \cite{EOD1001} con todo el contenido de desarrollado.
\item Software documentado.
\end{enumerate}


\capitulo{2}{Objetivos del proyecto}

"Este apartado explica de forma precisa y concisa cuales son los objetivos que se persiguen con la realización del proyecto. Se puede distinguir entre los objetivos marcados por los requisitos del software a construir y los objetivos de carácter técnico que plantea a la hora de llevar a la práctica el proyecto."

titulos:
	objetivos generales
		creacion de SE que sea capaz de conectarse en red tanto por wifi como por ethernet
		creacion de una relacion maestro esclavo con sistemas embebidos.
	objetivos técnicos
		uso ethernet
		uso wifi
		programacion software de sistemas empotrados
		programacion hardware de la placa
		control de placas esclavas desde placa maestro
		uso de comunicaciones serie
	objetivos personales
		comprender el funcionamiento de los sistemas embebidos y sus utilidades en la vida real
		entender el uso la tecnologia de la conectividad wifi y ethernet
		conocer el funcionamiento de envio y recepcion de paquetes
		extender mi conocimiento sobre los protocolos de internet TCP/IP, DHCP.
		ampliar 

\capitulo{3}{Conceptos teóricos}

En esta sección se detallarán los conceptos teóricos necesarios para comprender el desarrollo del proyecto. 

\section{Sistemas Embebidos}\label{sec:SE}

Ya hablamos anteriormente de que es un sistema empotrado, en este apartado profundizaremos más sobre ello, veremos las funcionalidades de estos dispositivos y su uso en un entorno real. También se detallarán los tipos de comunicación elegidos para este proyecto y el motivo de su elección en relación a otros pero, primero, ¿Qué es un sistema embebido?

Los sistemas embebidos o empotrados son herramientas de computación programadas con una o varias funcionalidades concretas. 
Las grandes ventajas de estos sistemas son que trabajan de forma autónoma, ininterrumpida y sin necesidad de mantenimiento. Estas características hacen que su uso sea muy interesante para el sector industrial y doméstico. Estos sistemas permiten hacer prácticamente cualquier tipo de tarea ya que además del hardware mínimo para que se ejecute un programa se le pueden añadir infinidad de periféricos que nutren de distintos usos a las placas más básicas. Al mismo tiempo el hecho de que se puedan comunicar entre ellas añade además varios usos muy interesantes, como podremos ver en el apartado de Trabajos Relacionados.
 
\subsection{Hardware}\label{sec:Hardware}

En los sistemas embebidos prácticamente todos los componentes están integrados en microcontrolador. En este caso el microcontrolador viene instalado en una placa de demostración que provee la opción de conectar periféricos mediante pines o entradas específicas para un periférico en concreto. 

\imagen{placa}{Placa FRDM-K64F}


En el caso de los sistemas embebidos cada uno de ellos se construye según el propósito especifico que se va a realizar con ello, es decir, dependiendo de su objetivo tendrá unas características hardware u otras. Sin embargo, sí que hay algunos componentes mínimos presentes en todas las placas como pueden ser por ejemplo, un microcontrolador (MCU) encargado de controlar las operaciones del SE. Un MCU está compuesto por un procesador, memoria Ram y Rom y puertos de entrada y salida. 
El MCU se encarga de ejecutar las instrucciones del programa cargado en memoria, en otras palabras, gestiona las entradas y salidas de datos. además de este componente, necesitaríamos de más periféricos para obtener algunas funcionalidades más específicas, algunos de estos periféricos son:

\begin{itemize}
\item[Puertos de comunicación]
\item[Sensores]
\item[Dispositivos de interfaz humana]
\item[Actuadores]
\item[Conversores ADC y DAC]
\item[Ultrasonidos]
\item[Puertos de comunicación]
\end{itemize}


\subsection{Software}

El software embebido o empotrado reside en memoria de sólo lectura. Con relación al software y hardware utilizados en este proyecto existen 2 posibilidades de cara a cargar el programa en el microcontrolador:
\begin{description}
\item[MBED] Este es el modo en el que vienen las placas por defecto. En este modo al conectar la placa al ordenador aparecerá como un medio extraíble y deberemos arrastrar los ficheros ".bin" en el que estaría el desarrollo de nuestro programa.
\item[OpenSDA] (Open Serial and Debug Adapter) o adaptador para depuración serie y comunicación serial en castellano. OpenSDA es la interfaz de bajo costo que ofrece NXP para la depuración y programación de sus
microcontroladores.
\end{description}

\imagen{openSDA}{Diagrama de Bloques de OpenSDA}


En el caso de este proyecto es necesario diferenciar algunos conceptos en lo relativo al software así como:

\begin{description}
\item[SO en tiempo real:\label{ref:SOTiempoReal}] Es un sistema operativo que se utiliza para facilitar la gestión de  multitareas y el uso de tareas en dispositivos con recursos y tiempos limitados. Además deben ser deterministas en el tiempo de ejecución. Para poder comprender adecuadamente la utilidad de un RTOS debemos conocer los siguientes conceptos:
\begin{description}
\item[Tarea] Las tareas, a las cuales también podríamos referirnos como procesos o hilos. Estos hilos se ejecutan de manera independiente, es decir, tienen su propio espacio de memoria. El aislamiento del espacio de memoria se garantiza mediante protección por hardware (MPU) restringiendo su acceso. Las tareas pueden tener diferentes estados según lo determine el RTOS:
	\begin{description}
		\item[Bloqueado:] La tarea está esperando un evento que puede ser un bloqueo de tipo mutex o semáforo, o una liberación de espacio en memoria .
		\item[Listo:] La tarea está lista para ejecutarse en la CPU, pero se mantiene a la espera porque la CPU ya está siendo utilizada.
		\item[En Ejecución:] La tarea se está llevando a cabo.
	\end{description}
\item[Programador:] Con programador nos referimos a la persona que desarrolla el funcionamiento del sistema operativo en tiempo real. Existen distintas técnicas para la programación de estos SO:
	\begin{description}
		\item[Expropiativo:] Se ejecuta la tarea con mayor prioridad. Incluso se puede interrumpir una tarea en curso si hay otra lista con prioridad mayor.
		\item[Apropiativo:] En este caso solo se interrumpe una tarea si esta aborta.
	\end{description}
En el caso de este proyecto FreeRTOS utiliza la programación Preventiva permitiendo así cumplir con las normas de tiempo real. Por otro lado esto tiene una mayor carga en el microcontrolador ya que tiene que gestionar el cambio de tarea.
\item[Comunicación entre tareas:] Como ha ocurrido en el software realizado es común que algunas variables locales de alguna tarea deban ser utilizadas en otras tareas. Para ello existen dos opciones. La primera y más simple es la utilización de variables globales. La segunda opción es la utilización de colas y buzones que nos modifiquen o lean el valor de esa variable.
\end{description}
\item[Middelware:] Este software se encarga de 'comunicar' el sistema operativo con los programas. Es un conjunto de librerías que sirven como rutinas para crear una infraestructura que ofrece servicios a los desarrolladores.
\item[Drivers:] Los \extranjerismo{drivers} o controladores de dispositivos proveen las instrucciones necesarias para que un dispositivo externo, un periférico, pueda ser controlado por el sistema operativo que el sistema embebido utilice.
\end{description}


\section{Tecnologías de Comunicación para SE}\label{sec:Comunicaciones}

Existen varias formas de comunicarse con los SE, tanto con componentes externos como entre dos o más microcontroladores. Por lo general, se pueden conectar los SE a periféricos que lo doten de conexiones \extranjerismo{bluetooth}, infrarrojos o wifi, entre otros. De esta manera conseguimos que podamos enviar y recibir información de otros sistemas. Por otro lado tenemos otro tipo de comunicaciones, en este caso para la comunicación con otros elementos, los periféricos. Según la funcionalidad que estemos usando se elegirán una tecnología acorde para el intercambio de datos. En los próximos dos apartados hablaremos sobre este tipo de tecnologías de comunicación y sus utilidades específicas. 

\subsection{Comunicación de Red}
En este apartado vamos a hablar sobre las conexiones de red. Por un lado, trataré la conexión vía ethernet que es la que se ha usado en el resultado final del proyecto. Por otro lado, también tocaré la conexión vía wifi ya que durante el desarrollo de este trabajo también se trató de utilizar este modo. Aunque, finalmente se acabó por elegir solo la conexión ethernet, se realizó bastante investigación sobre este tipo de conexión y como implementarla y puesto que si se consiguió transmitir datos mediante comandos 'AT' vía wifi, voy a exponer toda esta información en esta sección.

\subsubsection{TCP/IP}
Podríamos definir el modelo TCP/IP como un conjunto de normas que hace que varios equipos puedan comunicarse adecuadamente. Sería algo así como las normas sintácticas que seguimos los humanos para hablar y entendernos los unos a los otros. Este modelo es el más utilizado en Internet y se divide en cuatro grandes capas o niveles.

\imagen{protocolo-TCPIP}{Capas del modelo TCP/IP}

\begin{description}
\item[Capa de enlace de datos] 
Esta capa debe manejar las partes físicas del envío y recepción de datos. Esta comunicación se puede llevar a cabo mediante el cable Ethernet o de manera inalámbrica, la tarjeta de interfaz de red, el controlador del dispositivo en el equipo, etc.

\item[Capa de Internet] //MEJORAR////////
La capa de Internet (también denominada capa de red) controla el movimiento de los paquetes alrededor de la red. Se encarga del direccionamiento de los dispositivos y del empaquetado y manipulación de los datos para su correcto envío. Esta capa utiliza las versiones IPV4 e IPV6 para el control y notificación de errores, existe además, IGMP (Internet Group Management Protocol) y MLD (Multicast Listener Discovery) que se usan para establecer grupos de difusión múltiple.

\item[Capa de transporte]
Esta capa se asegura de conseguir que la conexión de datos sea fiable entre dos dispositivos. Para ello envía los datos en paquetes y se asegura de que el otro equipo indique que ha recibido los paquetes correctamente. En otras palabras, se encarga de facilitar la comunicación lógica entre dispositivos. Esta comunicación puede seguir dos protocolos:
	\begin{description}
	\item[TCP] es un protocolo orientado a la conexión, proporciona un conjunto completo de servicios para aquellas aplicaciones que lo necesiten y su uso se considera fiable. Con el uso de puertos consigue que varias aplicaciones puedan usar una misma dirección IP. El protocolo establece
una conexión virtual entre dos dispositivos capaz de enviar información de manera bidireccional. Las transmisiones usan una ventana deslizante que permite detectar aquellas no reconocidas para que
sean retransmitidas.
	\item[UDP] A diferencia de TCP, UDP no utiliza ningún mecanismo de establecimiento de la conexión, no se realizan retransmisiones, por lo tanto, determinadas transmisiones pueden llegar a perderse. Su uso se justifica en aquellos escenarios donde la velocidad de la transmisión prima por encima de todo aunque se incurra ocasionalmente en la perdida de información.
	\end{description}

\item[Capa de aplicación]
Esta capa nos ofrece la posibilidad de acceder a otras capas para usar sus servicios. Además a esta capa, pertenecen los protocolos como POP y SMTP que se utilizan para la comunicación, por ejemplo, el correo electrónico. Otro protocolo incluido en esta capa seria HTTP, que define la sintaxis y la semántica que se utiliza en la arquitectura web (clientes, servidores \extranjerismo{proxies}) para comunicarse. En este caso no es el usuario quien interactúa directamente con esta capa sino que interactúa con programas que, a su vez lo hacen con la capa de aplicación.
\end{description}



\subsection{Lwip}
La librería lwIP pretende dar un servicio basado en el protocolo TCP/IP. Este software fue desarrollado por Adam Dunkels en Computer and Laboratory de Arquitecturas de Redes (CNA) en el Instituto Sueco de Informática Ciencias (SICS).

El enfoque de la implementación de lwIP TCP/IP es reducir el uso de RAM sin dejar de tener un TCP a escala completa. Esto hace que lwIP sea adecuado para su uso en sistemas embebidos con decenas de kilobytes de RAM libre y espacio para alrededor de 40 kilobytes de código ROM.

\subparagraph{Características}
\begin{itemize}
  \item IP (Protocolo de Internet, IPv4 e IPv6), incluido el reenvío de paquetes
    múltiples interfaces de red
  \item ICMP (Protocolo de mensajes de control de Internet) para mantenimiento y depuración de redes
  \item IGMP (Protocolo de gestión de grupos de Internet) para la gestión del tráfico de multidifusión
  \item MLD (descubrimiento de oyentes de multidifusión para IPv6). Tiene como objetivo cumplir con
    RFC 2710. Sin soporte para MLDv2
  \item ND (descubrimiento de vecinos y configuración automática de direcciones sin estado para IPv6).
    Tiene como objetivo cumplir con RFC 4861 (descubrimiento de vecinos) y RFC 4862
    (Autoconfiguración de direcciones)
  \item UDP (Protocolo de datagramas de usuario) que incluye extensiones UDP-lite experimentales
  \item TCP (Protocolo de control de transmisión) con control de congestión, estimación de RTT
    y recuperación rápida/retransmisión rápida
  \item API nativa/sin formato para un rendimiento mejorado
  \item API de socket similar a Berkeley opcional
  \item DNS (resolución de nombres de dominio)
\end{itemize}

\subsection{Wifi: módulo ESP8266}

Según la informacion mostrada en \cite{moduloEsp8266} el módulo ESP8266 se trata de un chip integrado con conexión Wifi y compatible con el protocolo TCP/IP. El objetivo principal es dar acceso a cualquier microcontrolador a una red. La gran ventaja del ESP8266 es su bajo consumo. Soporta IPv4 y los protocolos TCP/UDP/HTTP/FTP.

\imagen{pines8266}{Pines del módulo wifi 8266}

\textbf{ESP32 soporta las siguientes características:}
\begin{itemize}
\item Soporta los principales buses de comunicación (SPI, I2C, UART).
\item Comunicación unicast encriptada y sin encriptar
\item Se pueden mezclar clientes con encriptación y sin encriptación
\item Permite enviar hasta 250-bytes de carga útil
\item Se pueden configurar \extranjerismo{callbacks} para informar a la aplicación si la transmisión fue correcta
\item Largo alcance, pudiendo superar los 200m en campo abierto.
\end{itemize}

\textbf{Pero también tiene sus limitaciones, las cuales son:}
\begin{itemize}
\item El número de clientes con encriptación está limitado. Esta limitación es de 10 clientes para el modo Estación, 6 como mucho en modo punto de acceso o modo mixto.
\item El número total de clientes con y sin encriptación es de 20.
\item Sólo se pueden enviar 250 bytes como mucho.
\end{itemize}

\section{FreeRTOS}\label{sec:RTOS}
Algunos apartados más atrás ya hablamos sobre que era un \extranjerismo{RTOS} \ref{ref:SOTiempoReal}, en esta sección veremos cual ha sido el sistema operativo que he utilizado en este proyecto.
FreeRTOS es robusto, tiene un tamaño reducido y prácticamente una compatibilidad del 100\%, por todo ello es el sistema operativo en tiempo real más utilizado en microcontroladores pequeños en el mundo.
Además cuenta con varias demostraciones preconfiguradas y detalladas referentes al Internet de las cosas (IoT). Esta mantenido por AWS en beneficio de la comunidad de FreeRTOS. Incluye, también, un soporte a largo plazo.
Como hemos podido observar es un software completísimo, referente mundial, fácil de usar y con un soporte que saca actualizaciones constantemente.


\subsection{Comunicaciones para los Periféricos}
Como ya mencionamos anteriormente la mayor funcionalidad que podemos dar a estas placas viene con la condición de poder comunicarnos entre ellas y con otros periféricos. Veamos algunos tipos de conexión, además de vía ethernet o wifi, que hemos utilizado en este trabajo.

\subsubsection{UART}
Uart, 'Universal Asynchronous Receiver-Transmitter' o en castellano Receptor-transmisor asíncrono universal. Su funcionamiento es sencillo, las conexiones son cruzadas entre los dos dispositivos, es decir el pin de transmisión (TX) del transmisor estará conectado al pin receptor (RX) del dispositivo que recibe la comunicación y viceversa. Esta comunicación es asíncrona por lo que no utiliza relojes para el envió y recepción de mensajes. Para que ambos dispositivos sepan cuando tienen que empezar y dejar leer bits se añaden a cada envío bits de inicio y parada. Es importante que ambos equipos tengan la misma tasa de Baudios configurada para que los bits se lean adecuadamente, de no ser así la comunicación fallaría ya que un dispositivo enviaría bits a una velocidad diferente de la que se leen ocasionando malas lecturas. La velocidad predeterminada suele ser de 115200 baudios.

Por otro lado tenemos la comunicación usart ‘Universal Synchronous and Asynchronous Receiver-Transmitter’ que puede realizar procesos de comunicación con relojes. 


\tablaSmallSinColores{Diferencias UART vs USART}{l c c}{personal}
{\multicolumn{1}{c}{Caracteristicas} & Usart & Uart\\}
{
Modo & Semidúplex & Dúplex\\
Velocidad & USART es mayor & UART es menor.\\
Funcionamiento & Señales de datos y de reloj & Señales de datos\\
Datos & Bloques	& Bytes \\
Complejidad & Más complejo & Más simple de utilizar\\ 
}

El principal motivo de hacerlo con Uart en vez de con Usart, es que los comandos de los motores se envían en bytes y no en bloques, por lo que, como hemos visto en la tabla, es necesario hacerlo por Uart.

\subsubsection{I2C}
El protocolo I2C, "Inter-Integrated Circuit" usa dos líneas para comunicarse con otros dispositivos. Por un lado tenemos SCL (línea de reloj en serie) y por otro lado tenemos SDA (línea de datos), esta última es la encargada de transmitir los datos. Este tipo de comunicación suele estar destinado al intercambio de datos con módulos o sensores y se usa en arquitecturas maestro-esclavo:
\begin{itemize}
\item Maestro: Dispositivo que proporciona un reloj para la comunicación
\item Esclavo: Dispositivo que utiliza el reloj del maestro
\end{itemize}
En nuestro caso utilizamos este tipo de comunicación para el uso de la pantalla LCD.

\section{Periféricos}\label{sec:perifericos}
En esta parte de la memoria vamos a ver las especificaciones de los periféricos utilizados en este proyecto

\subsection{potenciómetro y sensor de Temperatura}

El potenciómetro es un elemento hardware que busca determinar un potencial eléctrico en una conexión. Generalmente se consigue mediante la comparación de la potencia de entrada y la de salida. Esta diferencia de potencial es lo que se conoce como voltaje.
Su funcionamiento es sencillo, cuenta con tres resistencias, dos en cada uno de los extremos, que pueden cambiar de valor y una tercera resistencia con la cual podemos interactuar. Esta tercera resistencia nos permite aumentar o disminuir su resistencia propiamente dicha. De esta forma conseguimos poder variar el valor entre las conexiones.

Un sensor de temperatura o termómetro es un dispositivo que transforma los cambios de temperatura (magnitud física) en señales eléctricas(voltaje). Por lo general está  formado por un sensor encapsulado en una cubierta protectora. Entre el sensor y la capsula encontramos un material que es  conductor térmico y que permite traspasar rápidamente estos cambios de temperatura. Existen tres tipo de sensores de temperatura descritos perfectamente en \cite{TempSensTipos}:

\begin{description}
\item[Termopares] Funcionan mediante un principio de generación de una corriente entre dos metales diferentes unidos que tienen diferente comportamiento eléctrico en función de la temperatura. La señal generada se procesa y da lugar a una medición de temperatura. Son equipos sencillos, baratos y con una precisión suficiente para su uso en edificación. Sin embargo, tienen una respuesta lenta.
\item[Termorresistencias] Están constituidas por resistencias cuya conductividad varía en función de la temperatura, lo cual genera una señal que, una vez procesada permite obtener la medición de temperatura. Su velocidad de respuesta depende de la masa de la resistencia.
\item[Sensores electrónicos] Funcionan mediante dispositivos electrónicos que generan una corriente o señal en función de la temperatura. Son equipos con una respuesta mucho más rápida, pero más caros.
\end{description}


\capitulo{4}{Técnicas y herramientas}

En este capítulo se exponen todas las herramientas que se han utilizado y las técnicas que se han seguido para poder desarrollar el proyecto de una manera sencilla y organizada. 

\section{Técnicas: metodologías ágiles } \label{sec:HMetodologiasAgiles}
Como metodología ágil \cite{MetodologiasAgiles} para el desarrollo y organización del proyecto, se ha utilizado la metodología SCRUM, ya que además de ser la más usada en el entorno laboral real, ha sido una de las que hemos trabajado durante las asignaturas del grado. Pero para poder entender su potencial veamos que es SCRUM.

SCRUM es un proceso en el que se aplican un conjunto de buenas prácticas que permiten trabajar de forma colaborativa y organizada mediante el uso de Sprint y tareas. 
El desarrollo de todo el proyecto se separa en diferentes Sprint, que contienen una serie de tareas necesarias para conseguir un software final. Cada uno de los Sprint debe ser completado en unas 3 semanas por lo que en la reunión inicial, se deben discutir que tareas estarán en el sprint actual. El equipo deberá valorar la dificultad y el tiempo de cada una de las tareas y tratar de realizar la mayoría de las tareas posibles en ese sprint. En la finalización de cada sprint se deberá entregar un resultado válido. 
En la metodología SCRUM, es de gran importancia la comunicación con el resto del equipo, por lo que al final de cada día se realiza una reunión de unos 15 minutos en la que se exponen problemas, avances y dificultades. En la Figura \ref{diagSrum} encontramos un diagrama que sirve como resumen del funcionamiento de esta técnica.


%\imagen{SCRUM}{Diagrama resumen de la Metodología SCRUM. \cite{SCRUM}} \label{diagSrum}
\begin{figure}[!h]
	\centering
	\includegraphics[width=0.9\textwidth]{SCRUM}
	\caption{Diagrama resumen de la Metodología SCRUM. \cite{SCRUM}}\label{diagSrum}
\end{figure}

Las figuras más importantes de esta metodología son:
\begin{description}
\item[El Product Owner] Es el encargado de hablar con el cliente y exponer sus peticiones al SCRUM master y al equipo de desarrollo. En otras palabras, es quien define las tareas, condiciones y prioridades centrándose en el retorno sobre la inversión, en ingles Return on Investement(ROI), del proyecto.
\item[El SCRUM Master] Es quien dirige y lidera al SCRUM team. Es el encargado de guiar al equipo y hacer que se cumplan las normas de la metodología SCRUM. Esta figura está muy ligada al Product Owner, ayudándole a maximizar el ROI.
\item[El SCRUM team] Es el equipo desarrollador del software del proyecto. Suele estar formado por grupos de entre 3 y 9 personas que deben tener buenas capacidades referidas a la organización y gestión de tareas y procesos.
\end{description}

Partiendo de los actores anteriores podemos definir que, tanto el cargo de Product Owner y SCRUM Team han sido interpretados por mí, a diferencia de la figura de SCRUM Master la habrían representado mis tutores.

Todo esto nos ayuda a poder lograr un resultado final óptimo en el desarrollo de proyectos complejos, dónde se necesita obtener resultados en un corto margen de tiempo y donde las tareas pueden variar. Se usa en entornos complejos donde prima la competitividad, la flexibilidad y la productividad.


\section{Herramientas hardware}\label{sec:HHardware}

Veamos con más detalle los componentes hardware de este proyecto.

\subsection{Placa FRDM K64F}
Esta placa es un conjunto hardware que, además de contener el microcontrolador, nos ofrece las características necesarias para poder conectar varios periféricos a este controlador.

\begin{figure}[!h]
	\centering
	\includegraphics[width=0.7\textwidth]{placa}
	\caption{Placa FRDM K64F.}
\end{figure}
\FloatBarrier

Sus características son:
\begin{itemize}
\item Microcontrolador MK64FN1M0VLL12 MCU: (120 MHz, 1 MB de memoria flash, 256 KB RAM, USB de bajo consumo, sin cristal, y 100 Low profile Quad Flat (LQFP))
\item Núcleo ARM Cortex M4F. \cite{Cortex}
\item Interfaz USB de doble función con conector USB micro-B
\item LED RGB
\item Acelerómetro y magnetómetro FXOS8700CQ
\item Dos botones de usuario
\item Opción de alimentación flexible - OpenSDAv2 USB, Kinetis K64 USB y fuente externa
\item Fácil acceso a la entrada/salida del MCU a través de Arduino™ R3 compatible Conectores de E/S
\item Circuito de depuración programable OpenSDAv2 compatible con el software CMSISDAP Interface que proporciona:
\begin{itemize}
\item Interfaz de programación flash del dispositivo de almacenamiento masivo (MSD)
\item Interfaz de depuración CMSIS-DAP a través de una conexión USB HID sin controlador proporcionando depuración de control de ejecución y compatibilidad con herramientas IDE
\item Interfaz de puerto serie virtual
\item Proyecto de software CMSIS-DAP de código abierto
\item Ethernet
\item SDHC
\item Módulo RF adicional: nRF24L01+ Nordic 2.4GHz Radio
\item Módulo Bluetooth adicional: JY-MCU BT board V1.05 BT
\end{itemize}
\end{itemize}
 

Se puede encontrar más información acerca de los microcontroladores en el libro: \cite{Freescale}.

\subsection{Placa de expansión Arduino Basic I/O}

\begin{figure}[!h]
	\centering
	\includegraphics[width=0.6\textwidth]{shield}
	\caption{Placa de expansión Arduino basic I/O.}
\end{figure}
\FloatBarrier

Esta placa se sitúa sobre la placa K64F conectándola directamente sobre sus pines y dotándola de varias características, como por ejemplo:
Altavoz, 4 leds de diferentes colores, 4 botones, sensor de infrarrojos, 2 potenciómetros, entre otros. De esta manera se consigue dotar a la placa de más funciones.

\subsection{LCD}

Una pantalla LCD (\extranjerismo{liquid crystal display)} o pantalla de cristal líquido, es una pantalla delgada y plana. Puede estar formada por píxeles monócromos o en color, colocados sobre una fuente de luz. Estos píxeles son una capa de moléculas situada entre dos electrodos transparentes, dos filtros polarizados. Sin cristal líquido entre el filtro polarizante, la luz quedaría bloqueada al tratar de pasar el segundo filtro.

\begin{figure}[!h]
 \centering
  \subfloat[Pantalla LCD]{
    \includegraphics[width=0.5\textwidth]{PantallaLCD.png}}
  \subfloat[Modulo IIC/I2C]{
    \includegraphics[width=0.5\textwidth]{moduloLCD.png}}
 \caption{Pantalla LCD con modulo IIC/I2C.}
 \label{f:animales}
\end{figure}

Tanto para el desarrollador como sobre todo para el cliente final que recibe el proyecto, es interesante el uso de una pantalla que muestre información sobre las operaciones que está realizando el sistema empotrado. En este proyecto, se ha utilizado una pantalla de 2 líneas de 16 caracteres cada línea. Por lo general, estas pantallas necesitan de la conexión de más de 15 pines para la transmisión de datos, la alimentación, la iluminación y el control de la transmisión. Sin embargo, en este caso la pantalla utilizada incorpora un módulo I2C que deja la conexión en tan solo 4 pines que serían:
\begin{itemize}
\item[\textbf{SCL}] (\extranjerismo{System Clock}) es la línea de los pulsos de reloj que sincronizan el sistema. 
\item[\textbf{SDA}] (System Data) es la línea por la que se mueven los datos entre los dispositivos.
\item[\textbf{VCC}] Es el pin por el que reciben energía. En este caso serán 5 voltios.
\item[\textbf{GND}] Es el pin de tierra o masa que sirve para cerrar el circuito.
\end{itemize}
La comunicación utilizada en este tipo de elementos es I2C.


\subsection{Motores} 

El motor recibe el nombre de EMG30 (codificador, motor, reductor 30:1). Es un motor de corriente continua de 12v, totalmente equipado con codificadores y un reductor 30:1. 

\begin{figure}[!h]
	\centering
	\includegraphics[width=0.6\textwidth]{motorEMG30}
	\caption{Motor EMG30 utilizado en la planta piloto.}
\end{figure}
\FloatBarrier

Es ideal para aplicaciones robóticas pequeñas o medianas. También incluye un condensador de supresión de ruido estándar en el motor. Las conexiones del motor son:

\begin{enumerate}
\item Purple (1) Hall Sensor B Vout
\item Blue (2) Hall sensor A Vout
\item Green (3) Hall sensor ground
\item Brown (4) Hall sensor Vcc
\item Red (5) + Motor
\item Black (6) - Motor
\end{enumerate}

Estas conexiones van conectadas a la placa que se muestra en la Figura \ref{controlMotor} y será la encargada de controlar los bytes recibidos y enviados.

%\imagen{placaMotor}{Placa controladora de los motores} \label{controlMotor}
\begin{figure}[!h]
	\centering
	\includegraphics[width=0.9\textwidth]{placaMotor}
	\caption{Placa controladora de los motores.}\label{controlMotor}
\end{figure}

La placa K64F envía a través de conexión UART o I2C los bytes con las instrucciones para los motores a la placa controladora y posteriormente esta placa enviará la información al propio motor que realizará las acciones correspondientes a ese comando. 
Por otro lado, los bytes enviados consisten, en la mayoría de las instrucciones, en un byte de sincronización, un byte para elegir el motor, puesto que se pueden conectar hasta dos motores, y el byte con la instrucción. Estos bytes se pueden enviar tanto en formato hexadecimal como decimal u octal.
Vamos a ver los modos y comandos que se pueden utilizar con estos motores.
En cuanto a los modos disponemos de 4 modos:
\begin{itemize}
\item Modo 0. Si se elegimos usar el modo 0 entonces los registros de velocidad son velocidades literales en el rango de 0 (retroceso total)
128 (parada) 255 (avance total).
\item Modo 1. El modo 1 es similar al modo 0, excepto que los valores de velocidad se interpretan como valores con signo. El rango es -128
(retroceso total) 0 (parada) 127 (avance total).
\item Modo 2. En el modo 2 la velocidad 1 controle la velocidad de ambos motores, y la velocidad2 se convierte en el valor de giro. Los datos están en el rango de 0 (retroceso total) 128 (parada) 255 (avance total).
\item Modo 3. El modo 3 es similar al modo 2, excepto que los valores de velocidad se interpretan como valores con signo. Los datos están en el rango de -128 (retroceso total) 0 (parada) 127 (avance total).
\end{itemize}

En la Tabla \ref{tabla:ComandosMotores} se muestran los comandos (CMD) más importantes y su descripción.

\tablaSmallSinColores{Comandos Motores EMG30.}{l c c l}{ComandosMotores}
{\multicolumn{1}{c}{ CMD } & Nombre & Bytes & Descripción\\ 
 & & Env-Rec & \\}
{
0x21 & Get Speed 1 & 2 - 1 & Obtener la velocidad del motor A\\
0x22 & Get Speed 1 & 2 - 1 & Obtener la velocidad del motor B\\
0x2A & Get Acceleration & 2 - 1 & Devuelve la aceleración\\
0x31 & Set Speed 1 & 3 - 0 & Fija la velocidad del motor A\\
0x32 & Set Speed 1 & 3 - 0 & Fija la velocidad del motor B\\
0x33 & Set Acceleration & 3 - 0 & Fija la aceleración\\
0x34 & Set Mode & 3 - 0 & Fija el modo\\ 
0x38 & TimeOut OFF & 2 - 0 & No apagar en 2s sin comunicación\\
0x39 & TimeOut ON & 2 - 0 & Apagar tras 2s sin comunicación\\
}

\subsection{Sensor de Temperatura: LM35}

\begin{figure}[!h]
	\centering
	\includegraphics[width=0.6\textwidth]{Tmp}
	\caption{Sensor de temperatura, LM35.}
\end{figure}
\FloatBarrier

El sensor de temperatura LM35, es un circuito electrónico cuyas características se describieron en el capitulo anterior \ref{potenySensorTemp}, de salida analógica, que permite medir temperaturas. Este sensor proporciona un voltaje proporcional a la temperatura en la que se encuentra. Tiene un rango de medición desde -55 grados centígrados, hasta un máximo de 150 grados.
Sus características principales son:
\begin{itemize}
\item Resolución: 10mV por grado centígrado.
\item Voltaje de alimentación.  Por ejemplo, este sensor se puede alimentar desde 4Vdc hasta 20Vdc.
\item Tipo de medición: Salida analógica.
\item Numero de pines: 3 pines, GND, VCC y VSalida.
\end{itemize}


\subsection{Wifi: módulo ESP8266}

El módulo ESP8266 \cite{moduloEsp8266} se trata de un chip integrado con conexión Wifi y compatible con el protocolo TCP/IP. El objetivo principal es dar acceso a cualquier microcontrolador a una red. La gran ventaja del ESP8266 es su bajo consumo. Soporta IPv4 y los protocolos TCP/UDP/HTTP/FTP. La Figura \ref{pinesESP} muestra los pines del módulo ESP8266.

%\imagen{pines8266}{Pines del módulo wifi 8266}  \label{pinesESP}
\begin{figure}[!h]
	\centering
	\includegraphics[width=0.9\textwidth]{pines8266}
	\caption{Pines del módulo wifi 8266.}\label{pinesESP}
\end{figure}

\textbf{ESP32 soporta las siguientes características:}
\begin{itemize}
\item Soporta los principales buses de comunicación (SPI, I2C, UART).
\item Comunicación unicast encriptada y sin encriptar.
\item Se pueden mezclar clientes con encriptación y sin encriptación.
\item Permite enviar hasta 250 bytes de carga útil.
\item Se pueden configurar \extranjerismo{callbacks} para informar a la aplicación si la transmisión fue correcta.
\item Largo alcance, pudiendo superar los 200 metros en campo abierto.
\end{itemize}

\textbf{Pero también tiene sus limitaciones:}
\begin{itemize}
\item El número de clientes con encriptación está limitado. Esta limitación es de 10 clientes para el modo Estación, 6 como mucho en modo punto de acceso o modo mixto.
\item El número total de clientes con y sin encriptación es de 20.
\item Sólo se pueden enviar hasta 250 bytes.
\end{itemize}

\subsubsection{Uso del módulo ESP8266}
El módulo ESP8266 \cite{tutorialESP} es un dispositivo TTL ``Serial to Wireless Internet'' funciona mediante el envío de comandos `AT'. En la Tabla \ref{tabla:ComandosAT} se explican los comandos más importantes. Los comandos están ordenados según los pasos que se deberían seguir para establecer comunicación entre un módulo y un servidor.

\clearpage

\tablaSmallSinColores{Comandos AT.}{l l}{ComandosAT}
{\multicolumn{1}{c}{Comando} & Descripción\\}
{
AT+CWMODE=`X' & \parbox{.5\textwidth}{\begin{itemize}
  	\item 1 = Modo estación (cliente)
	\item 2 = Modo AP (huésped)
	\item 3 = Modo AP + Estación (modo dual)
    \end{itemize}}\\ \hline \\
AT+CWLAP & Lista APs disponibles \\ \hline \\
AT+CWJAP=`ssid',`Pswd' & \parbox{.5\textwidth}{El módulo se conecta a la red con el nombre ssid indicado y la contraseña pwd suministrada.} \\ \hline \\
AT+CIPMUX=`X' & \parbox{.5\textwidth}{\begin{itemize}
  	\item 0 = Conexión única
	\item 1 = Múltiples conexiones, hasta 4
    \end{itemize}}\\ \hline \\
AT+CIPSERVER=Mode,puerto & \parbox{.5\textwidth}{
Configura el módulo como servidor donde el
modo:	
	\begin{itemize}
  	\item 0 = Borrar servidor
	\item 1 = Crear servidor
    \end{itemize}
    puerto: número del puerto, por defecto es el 333}\\ \hline \\
AT+CIFSR & \parbox{.5\textwidth}{Retorna la dirección IP local del módulo como cliente.} \\ \hline \\
AT+CIPSEND & \parbox{.5\textwidth}{Envía datos sin adornos cada 20ms. El módulo retorna ">" después ejecutar el comando, si se recibe el comando "+++" se regresa al modo comando.}\\
}

\section{Herramientas software}\label{sec:HSoftware}

Veamos que herramientas software he utilizado a lo largo del desarrollo del proyecto.

\begin{description}
\item[KDS IDE.]
Kinetis ® Design Studio (KDS) es un entorno de desarrollo integrado complementario para los MCU Kinetis que permite una edición, compilación y depuración sólidas de sus diseños.
Kinetis está basado en software gratuito de código abierto que incluye Eclipse, GNU Compiler Collection (GCC), GNU Debugger (GDB) y otros, Kinetis Design Studio IDE ofrece a los diseñadores una herramienta de desarrollo simple sin limitaciones de tamaño de código. 
Además, el software Processor Expert ® habilita su diseño con su base de conocimiento y ayuda a crear aplicaciones potentes con unos pocos clics.

Este fue el IDE sobre el que empecé el proyecto y del que tras un par de semanas terminaría migrando a MCUXPRESSO 
Entorno multiplataforma basado en software libre como Eclipse IDE o GNU Compiler Collection (GCC). Incorpora Processor Expert, una utilidad que permite añadir y configurar los componentes necesarios para un proyecto.


\item[MCUXpresso IDE.]
Este IDE está basado en eclipse y sobre el que se ha desarrollado el proyecto. 
El IDE de MCUXpresso \cite{MCUXDownload} ofrece funciones avanzadas de edición, compilación y depuración. Añade también vistas de depuración específicas de MCU, rastreo y creación de perfiles de código, además de depuración multinúcleo y herramientas de configuración integradas. Es un IDE muy completo que presenta una interfaz simple para el usuario pese a gran número de opciones y configuraciones que ofrece.

\section{FreeRTOS}\label{sec:RTOS}
Algunos apartados más atrás ya hablamos sobre que era un \extranjerismo{RTOS} \ref{ref:SOTiempoReal}, en esta sección veremos cual ha sido el sistema operativo que he utilizado en este proyecto. \\
FreeRTOS \cite{web:FreeRtos} es robusto, tiene un tamaño reducido y  una compatibilidad del 100\% con el microcontrolador. Es el sistema operativo en tiempo real más utilizado en microcontroladores pequeños en el mundo.\\
Además cuenta con varias demostraciones preconfiguradas y detalladas referentes al Internet de las cosas (IoT). AWS se encarga de su mantenimiento y soporte a largo plazo.
Como hemos podido observar es un software completísimo, referente mundial, fácil de usar y con un soporte que saca actualizaciones constantemente.


\subsection{LwIP}
La librería light weight IP pretende dar un servicio basado en el protocolo TCP/IP. Este software fue desarrollado por Adam Dunkels en Computer and Laboratory de Arquitecturas de Redes (CNA) en el Instituto Sueco de Ciencias de la Computación (SICS).

El enfoque de la implementación de lwIP TCP/IP es reducir el uso de RAM sin dejar de tener un TCP a escala completa. Esto hace que lwIP sea adecuado para su uso en sistemas embebidos con decenas de kilobytes de RAM libre y espacio para alrededor de 40 kilobytes de código ROM.

\subparagraph{Protocolos implementados}
\begin{itemize}
  \item IP (Protocolo de Internet, IPv4 e IPv6), incluido el reenvío de paquetes múltiples interfaces de red.
  \item ICMP (Protocolo de mensajes de control de Internet) para mantenimiento y depuración de redes.
  \item IGMP (Protocolo de gestión de grupos de Internet) para la gestión del tráfico de multidifusión.
  \item MLD (descubrimiento de oyentes de multidifusión para IPv6). Tiene como objetivo cumplir con RFC 2710. Sin soporte para MLDv2.
  \item ND (descubrimiento de vecinos y configuración automática de direcciones sin estado para IPv6).
    Tiene como objetivo cumplir con RFC 4861 (descubrimiento de vecinos) y RFC 4862
    (Autoconfiguración de direcciones).
  \item UDP (Protocolo de datagramas de usuario) que incluye extensiones UDP-lite experimentales.
  \item TCP (Protocolo de control de transmisión) con control de congestión, estimación de RTT y recuperación rápida/retransmisión rápida.
  \item API nativa/sin formato para un rendimiento mejorado.
  \item API de socket similar a Berkeley opcional.
  \item DNS (resolución de nombres de dominio).
\end{itemize}

\item[Dockligth.]
Docklight \cite{DocklightDownload} es una herramienta de prueba, análisis y simulación de protocolos de comunicación en serie.
Este programa se utiliza para captar las comunicaciones serie y Uart. En este proyecto se ha utilizado para poder comunicarnos con la placa de una manera más sencilla a la hora de tener que introducir comandos. Para la captura de estos mensajes es necesario conocer el puerto de salida 'comm' y la velocidad en baudios a la que se transmiten los datos, además del número de bits, paridad, etc.
Por otro lado, este software cuenta con algunas características añadidas como poder guardar comandos que usamos de forma continua o poder ver la información en ascii, binario o hexadecimal que en algunas ocasiones puede ser necesario.

\item[Termite.]
Este software es muy parecido a Docklight pero en este caso cuenta con una interfaz más simple. En este caso el programa se configura con los parámetros del otro dispositivo con el que nos vamos a comunicar y se reciben o envían datos.

\item[Packet Sender.]
Este programa ha sido de gran ayuda puesto que se utilizó para realizar las pruebas de recepción y envío de paquetes a las tres placas. 
Packet Sender es una utilidad de código abierto que permite enviar y recibir paquetes TCP y UDP. También admite conexiones TCP mediante SSL, generación de tráfico intenso, solicitudes HTTP GET/POST y generación de paneles.
\end{description}


\section{Herramientas de documentación}\label{sec:HDocumentacion}
Veamos las herramientas utilizadas para documentar y trabajar sobre el proyecto.

\begin{description}
\item[Textmaker. \cite{wiki:TextMaker}]
Es una herramienta gratuita que nos ayuda a escribir documentos de texto integrando las funciones necesarias para poder realizar documentos con Latex. Además este software es multiplataforma.
\item[Latex. \cite{wiki:latex}]
Latex es un compositor de textos destinado a la creación de documentos profesionales que requieran una alta calidad tipográfica. Se utiliza, por lo general, en la realización de artículos y libros científicos que incluyen elementos y expresiones matemáticas.
\item[Bibtex. \cite{bibtex}]
Es una herramienta utilizada para la creación de referencias bibliográficas. Genera un formato para cada una de las referencias con los datos aportados por el usuario y generalmente se utiliza en la realización de documentos con LaTeX.
\end{description}

\section{Herramientas de comunicación}\label{sec:HComunicacion}
Para la comunicación con mis tutores para aclarar dudas y resolver fallos se han utilizado las siguientes herramientas.

\begin{description}
\item[Microsoft Outlook. \cite{wikiOutlook}] Es un gestor de correo electrónico desarrollado por Microsoft y que podemos encontrar en la suite de Microsoft office.

\item[Microsoft Teams. \cite{wikiTeams}] De nuevo es un programa desarrollado por Microsoft. En este caso se trata de una plataforma para realizar reuniones virtuales, cuenta también con salas de chat y la posibilidad de generar documentos compartidos.
\end{description}


\section{Herramientas de gestión de proyectos}\label{sec:HGestionProyectos}
En el primer apartado hablábamos de la técnica de organización y desarrollo de proyectos mediante metodologías ágiles: SCRUM, en este apartado veremos cuales son las herramientas con las que conseguimos facilitar estas tareas.

\begin{description}
\item[GitHub. \cite{GitHub}]
GitHub es una plataforma pensada para que los desarrolladores puedan alojar su repositorios de código de forma segura en la nube. Además incluye un sistema de control de versiones conocido como Git. \\
Por otro lado, también permite el desarrollo colaborativo entre distintos desarrolladores y ofrece todas las herramientas para poder trabajar con SCRUM. 

\item[GitKraken. \cite{GitKraken}]
GitKraken es una aplicación que nos permite manejar Git y por tanto nuestros archivos de GitHub de forma más sencilla. Esta herramienta se encuentra disponible para todas las plataformas.
\end{description}










\capitulo{5}{Aspectos relevantes del desarrollo del proyecto}

A medida que avanzaba con el proyecto han ido surgiendo algunos hitos importantes que voy a remarcar en este capítulo.

\section{IDE: Kinetis vs MCUXpresso}\label{sec:ARKinetisvsMCUX}
Como ya vimos en el apartado de herramientas se han utilizado dos \extranjerismo{IDEs}, ambos dos creados por la misma empresa, NXP. 
En un primer momento se comenzó utilizando Kinetis Design Studio puesto que mis tutores me facilitaron algunas prácticas realizadas con ese programa de cara a familiarizarme con ello. Tras la realización de las prácticas enseguida me di cuenta de que estaba bastante anticuado y que las nuevas versiones de \extranjerismo{drivers} y \extranjerismo{middelware} no funcionaban correctamente.
Es por ello que decidí migrar hacia MCUXpresso. Este cambio hizo que tuviera que volver a familiarizarme con el nuevo IDE y la manera de programar el hardware a bajo nivel con la herramienta \extranjerismo{Config Tools}. Aunque KDS estuviera anticuado, en lo referente al uso de esta herramienta era algo más sencillo. A favor de la MCUX diré que incluye decenas de programas de ejemplo con los que poder probar y comprender su funcionamiento, además de un montón de \extranjerismo{drivers} que habilitan la instalación de un gran número de sensores. También se puede observar que MCUXpresso cuida más la limpieza y simplicidad de las interfaces a la hora de interactuar con ella. Una vez comprendes como funciona la programación de periféricos, relojes y pines, lo cual no es sencillo en un principio, programar un sistema embebido se convierte en algo sencillo.

\section{FRDM K64F vs Arduino}\label{sec:ARK64FvsArduino}

Este proyecto también podría haberse desarrollado con las placas Arduino UNO, las cuales son muy parecidas en cuanto a su funcionamiento y uso de periféricos. Además, utilizar Arduino hubiera sido más sencillo debido a que los pines vienen ya configurados, al igual que los relojes y tan solo hubiera sido enchufar y programar las funciones de las tareas que quisiéramos. Arduino también cuenta con una comunidad mayor por lo que tendríamos librerías más simples y avanzadas y mayor información en Internet de cara a resolver fallos. Entonces, ¿Por qué decidí utilizar las FRDM K64F?\\
La respuesta es simple, el objetivo de la realización de este trabajo no era simplemente desarrollar una planta piloto que realizara una tarea útil. El objetivo principal no es que los motores funcionen adecuadamente, o la pantalla, o el sensor de temperatura, etc. El objetivo principal era aprender sobre el funcionamiento de los sistemas embebidos. Es por ello que se decidieron utilizar estas placas, `puesto que ofrece al usuario mayor control sobre ella por la necesidad de ser programada a bajo nivel.
Desde un principio se ha pretendido centrar los esfuerzos en comprender como se realiza la configuración de sus pines y relojes, etc. En un entorno real, se deben conocer el funcionamiento de estos sistemas a bajo nivel puesto que, cada proyecto desarrollado en el entorno laboral necesitará que el SE cumpla con unos requisitos específicos y no exceda de ellos para disminuir el coste y tamaño del sistema. En cada proyecto se desarrollan unas placas específicas para esos requisitos y es conveniente saber su funcionamiento interno para poder diseñarlas correctamente.\\
Por otro lado, en el caso del homólogo a la placa FRDM K64F, que sería el Arduino Uno, no hubieramos podido utilizar ni FreeRTOS puesto que no dispone de reloj de tiempo real, ni lwIP debido a que su potencia es menor a la de las placas K64F. Además, con Arduino tampoco hubieramos podido \extranjerismo{debuguear} puesto que no usa OpenSDA para cargar el software. 
Debido a todo esto se optó por la utilización de las placas FRDM-K64F.


\section{RTOS}\label{sec:ARRTOS}
En el caso de los sistemas operativos en tiempo real encontramos, algunas alternativas a FreeRTOS. Las características y conceptos más importantes de FreeRTOS ya los vimos en el apartado \ref{sec:RTOS}. Como alternativa a este sistema operativo encontramos: 
\begin{description}
\item \extranjerismo{Embedded Operating System} (embOS), es un sistema operativo en tiempo real, desarrollado por la empresa SEGGER Microcontroller. Está diseñado para ser utilizado como base para el desarrollo de aplicaciones integradas en tiempo real para una amplia gama de microcontroladores. El funcionamiento es prácticamente el mismo.
\item MQX es otra opción de sistema operativo en tiempo real propuesto por NXP. Es un SO que ofrece una API sencilla y una arquitectura modular que hace que este software sea escalable.
\end{description}
El motivo de haber elegido la utilización de FreeRTOS es su presencia en un mayor número de proyectos que sus competidores. Esto hace que existan comunidades en Internet, que nos brindan mayor información y soluciones para su correcta utilización. Además de que su propio manual ya nos ofrece los pasos a seguir, es realmente sencillo de utilizar, una vez has aprendido los conocimientos básicos de su funcionamiento.

\section{Metodologías Ágiles}\label{sec:ARMetodologiasAgiles}

Como alternativas a SCRUM teníamos dos metodologías ágiles muy utilizadas y quizás más sencillas de implementar:

\begin{description}
\item[Extreme Programming XP]
Esta metodología es muy utilizada en pequeñas empresas durante sus comienzos y consolidación. Se basa en centrar sus esfuerzos en la comunicación entre clientes y empleados, potenciando las relaciones personales mediante el trabajo en equipo y promoviendo la comunicación y la eliminación de tiempos muertos.
Sus principales fases son:
\begin{enumerate}
\item Diseño y planificación del proyecto con el cliente.
\item Programación por parejas dentro del equipo de forma que ambos puedan intercambiar conocimientos consiguiendo mejores resultados.
\item Realización continua de pruebas de código.
\end{enumerate}

\item[Kanban] 
La metodología Kanban se basa en el uso de un \extranjerismo{layout} con columnas, generalmente tres, en las que se muestran las tareas que quedan por hacer: 'Pendientes', las que están en curso: 'En proceso' y las que ya se terminaron: 'Terminadas'. Cada usuario o equipo tiene la posibilidad de aumentar el número de columnas según les sea de mayor utilidad. De esta manera se tiene conocimiento sobre el estado del proyecto en tiempo real, mejorando la productividad y eficiencia del desarrollo del trabajo.
Las principales ventajas de esta metodología son:
\begin{enumerate}
\item Facilidad para la planificación de tareas.
\item Mejora en el rendimiento de trabajo del equipo.
\item Visión global de un solo vistazo.
\item Los plazos de entregas son continuos.
\end{enumerate}
\end{description}

Pese a estas dos alternativas se decidió usar SCRUM ya que es la metodología más usada y estudiada durante el grado. Además, el uso de la herramienta GitHub hace que sea fácil de usar y también consigue que la generación de código quede bien expuesta y organizada.

\section{Dificultades y Problemas}\label{sec:ARDificultades}
Durante el desarrollo de todo el proyecto he tenido algunos inconvenientes tanto personales como técnicos. Veamos algunos de ellos:
\begin{description}
\item Ya desde un principio perdí tiempo por el cambio de IDE y tener que familiarizarme de nuevo con las interfaces del programa. 
\item Por otro lado, la parte de configuración de pines, periféricos y relojes es algo complicada, sobre todo al principio del proyecto, ya que durante el grado no se ha visto nada parecido. A la hora de buscar información sobre como realizar algún procedimiento, tanto en código, como en configuración de pines, no encontraba demasiada información. El uso de estas placas es algo específico ya que están pensadas para usuarios con ciertos conocimientos en el ámbito de los SE. 
\item Relacionado con la organización y planificación fue complicado puesto que mi situación personal fue cambiando a lo largo del proyecto. Al principio de su realización no tenia demasiadas obligaciones, lo que me permitía tener un cierto control en cuanto a horas y horarios diarios. A mitad del desarrollo comencé a realizar las prácticas curriculares y el tiempo libre disminuye complicando tener una estabilidad diaria para realizar este proyecto.
\item Volviendo al desarrollo, la adaptación del tipo de comunicaciones, sobre todo la UART, al envío de los comandos de los motores, fue algo compleja. Estas dificultades vinieron dadas en gran parte por los cambios de tipos.
\end{description}

\section{Caso de uso Real}
El ultimo aspecto relevante de este proyecto que quiero destacar es el uso de este proyecto en un entorno real. Durante el tiempo que he estado realizando el TFG también he estado cursando las prácticas curriculares en la empresa Kronospan S.L. Esta empresa me propuso poder usar parte del proyecto en sus instalaciones, aunque solo fuera a modo de representación de un caso de uso real. Por ello decidimos que una gran utilidad de estos sistemas para la empresa y departamento en el que estaba, podía ser la medición de la temperatura de su CPD en tiempo real. Para desempeñar esta tarea se puso una placa con el sensor de temperatura dentro del CPD conectada en red por cable y por otro lado se puso otro SE con una pantalla LCD y un led rojo en la mesa en la que trabajo. El sensor de temperatura enviaba la temperatura actual si pasaba de un máximo determinado y el SE de mi mesa reportaba la temperatura con el led rojo y la pantalla que mostraba la temperatura en ese momento. Esto se utilizó durante una semana a forma de demostración, por supuesto al tratarse de una multinacional ellos ya disponían de sus propios sensores en cajas de Rittal con sistemas anti incendios, varios sistemas de aire acondicionados, etc. A continuación, se muestran algunas imágenes de la ejecución de ese proyecto.

\begin{figure}[!h]
 \centering
  \subfloat[Temp: 31 grados]{
    \includegraphics[width=0.4\textwidth]{CU31.png}}
  \subfloat[Temp: 34 grados]{
    \includegraphics[width=0.475\textwidth]{CU34.png}}
 \caption{Caso de uso Kronospan.}
\end{figure}

\clearpage

En la Figura \ref{CPD} vemos como la placa maestro está introducida en el CPD para la captación de la temperatura.
%\imagen{CUCPD}{Placa `Maestro' en el CPD.} \label{CPD}
\begin{figure}[!h]
	\centering
	\includegraphics[width=0.9\textwidth]{CUCPD}
	\caption{Placa `Maestro' en el CPD.}\label{CPD}
\end{figure}

Estos han sido los hitos que más dificultades me han causado durante la realización del TFG.










\capitulo{6}{Trabajos relacionados}\label{cap:trabajosRelacionados}

Este capítulo mostrara algunos trabajos que se han hecho con sistemas empotrados. De esta manera podremos hacernos a la idea de algunas de las aplicaciones reales de estos sistemas. Veamos los distintos sectores en los que se utilizan los SE:

\section{SE en equipos médicos}\label{sec:TREquiposMedicos}

Muchos de los aparatos que se utilizan de forma periódica en la sanidad, usan un microcontrolador \cite{medicinaSE}. La Figura \ref{PresionSanguinea} muestra un ejemplo de un medidor de presión sanguínea:

%\imagen{presionSangre}{Esquema arquitectónico de un SE de medición de sangre.} \label{PresionSanguinea}
\begin{figure}[!h]
	\centering
	\includegraphics[width=0.9\textwidth]{presionSangre}
	\caption{Esquema arquitectónico de un SE de medición de sangre.}\label{PresionSanguinea}
\end{figure}

En la imagen podemos apreciar como existen dos partes importantes, por un lado tenemos todos los elementos que necesitan corriente y cómo se conectan a una fuente de alimentación y por otro lado está el microcontrolador al que se conectan todos los periféricos que nos ayudan a llevar a cabo la tarea que queremos realizar. Algunos de esos periféricos son un altavoz o una pantalla para poder interactuar con el usuario que use el sistema. También tenemos el sensor de presión sanguínea que utilizará un valor analógico y lo convertirá en uno digital (ADC) para poder saber la presión sanguínea del cliente.
Otro ejemplo sería la utilización de un SE para controlar un desfibrilador (DESA). Un desfibrilador sirve para recuperar el ritmo cardíaco. En la actualidad se pueden encontrar este tipo de sistema de primeros auxilios, en algunos establecimientos, generalmente en ambientes deportivos o donde el flujo de gente es de avanzada edad. Estos aparatos cuentan con un microcontrolador que mediante un altavoz explica como debes usarlo y aplica de forma cíclica una serie de descargas dependiendo del estado del paciente y con poca interacción humana. En este caso el esquema arquitectónico sería muy parecido a la figura que presentaba arriba cambiando el flujo de aire por un sensor de ritmo cardíaco. 

\section{SE en gestión de la energía}\label{sec:TRGestionEnergetica}

Todos en nuestra casa disponemos de una caldera y un termostato con el que controlamos la temperatura. Este termostato podría ser perfectamente un sistema empotrado. En este ejemplo además vamos a poder observar la importancia y utilidad de que los SE puedan comunicarse entre ellos y no solo con otros periféricos. Veamos la siguiente Figura \ref{CalefaccionSE}:

%\imagen{seCalefaccion}{Sistema de Calefaccion con SE.} \label{CalefaccionSE}
\begin{figure}[!h]
	\centering
	\includegraphics[width=0.9\textwidth]{seCalefaccion}
	\caption{Sistema de Calefaccion con SE.}\label{CalefaccionSE}
\end{figure}

Como podemos apreciar en la Figura \ref{CalefaccionSE} se muestran 2 sistemas embebidos, además de uno por cada zona a calentar. El primer sistema embebido, `SE1', es el que utilizará el usuario para encender la caldera y configurar la temperatura de cada zona. El SE1 se comunica con la caldera y con el `SE2'. El SE2 es el encargado de abrir o cerrar las electroválvulas que dejaran pasar el agua caliente a cada una de las zonas calentándolas. El SE2 se comunicará con cada uno de los sistemas empotrados que encontramos en cada zona para conocer la temperatura a la que están. 
Como hemos visto de esta manera podremos conseguir distintas temperaturas en cada zona dependiendo de nuestras necesidades sin necesidad de tener que subir la calefacción de todas las habitaciones de la casa o de todas las oficinas de la empresa en caso de que alguna, por ejemplo, no se esté usando en ese momento.

\section{Conexión desde otros dispositivos}\label{sec:TRConexiones}
En este apartado me gustaría hacer mención del trabajo de fin de máster que desarrolló un compañero, \cite{RPC0027}
, de esta misma universidad en el año 2018. Su TFG estaba relacionado con los sistemas empotrados. En este caso, él realizó un la conexión a Internet de una sola placa que se controlaba a través de un servidor que el gestionaba. Utilizando las interfaces propuestas en el servidor podía gestionar las luces led de la placa, tanto su color como intensidad. Me parece muy interesante la idea de poder controlar estas placas ya no solo mediante sus botones, potenciómetros, etc, sino el hecho de poder hacerlo desde un ordenador o dispositivo móvil. 
Esto es una gran idea por varias razones:
\begin{itemize}
\item En primer lugar no necesitamos de un SE que nos haga de `puente' para poder configurar las acciones de otros puesto que utilizamos un móvil u ordenador, elementos que siempre tenemos a mano. 
\item Además de esta manera conseguimos poder comunicarnos con las otras placas desde cualquier sitio y no desde donde esté físicamente ese sistema.
\end{itemize}
Roberto lo hizo, en este caso, mediante una conexión a Internet, pensando en poder comunicarnos con él aunque estemos a grandes distancias pero como se comentó en los primeros capítulos existen más formas en caso de que lo tengamos relativamente cerca, como serían bluetooth o incluso radiofrecuencias.



\capitulo{7}{Conclusiones y Líneas de trabajo futuras}

En este capítulo veremos las conclusiones llegadas tras haber terminado el proyecto. Además se expondrán algunas ideas de cómo podríamos continuar con este mismo trabajo o realizar algunos distintos también basados en los sistemas embebidos y las comunicaciones entre ellos.


\section{Conclusiones} \label{sec:Conclusiones}
En este proyecto se han trabajado conocimientos adquiridos en varias asignaturas entre las cuales destacarían:

\begin{itemize}
\item Programación concurrente y de tiempo real.
\item Administración de redes y sistemas.
\item Programación.
\item Gestión de proyectos.
\end{itemize}

Por supuesto, este trabajo también me ha hecho darme cuenta de muchos conocimientos que no tengo y he tenido que aprender.
Otra de las partes más importantes de la realización del TFG es la gestión emocional. El hecho de saber organizarte para un trabajo de tantas horas con una fecha de entrega tan lejana, junto con la realización de las demás actividades de tu vida, hacen que tengas que mejorar tus habilidades en materias como la responsabilidad, organización, fuerza de voluntad, perseverancia, entre otras, y por supuesto al mismo tiempo tienes que luchar contra otras muchos vicios contraproducente tales como la pereza y tener que sobreponerte a las adversidades. 
En cuanto a la organización y planificación del proyecto ha sido muy enriquecedor el hecho de utilizar GitHub y la metodología Scrum. De esta manera he podido ver cómo se realiza un proyecto de este tipo en un ejemplo real y aprender cómo funciona esta herramienta. Además usarla ha permitido que en caso de fallos en la programación del sistema embebido dispusiera siempre de una copia anterior que me servía como \extranjerismo{backUp}.

Dicho esto es importante hacer hincapié en que, habiendo terminado el trabajo puedo dar por completados los objetivos técnicos \ref{sec:OTecnicos}, generales \ref{sec:OGenerales} y personales \ref{sec:OPersonales}. 

Personalmente ha sido de gran interés el uso de sistemas embebidos en mi proyecto, puesto que a medida que me adentraba más en ese campo y comprendía su utilización cada vez surgían más y más usos que se le pueden dar, tal y como veremos en el próximo apartado, 'líneas de trabajo futuras'. Además su estructura permite ampliar sus funcionalidades fácilmente por lo que cada vez se utilizan más en distintos ámbitos como, industrial, electrónico, informático o incluso en la salud. 
Otra de las grandes ventajas de usar estos sistemas en la actualidad es el hecho de que se puedan comunicar entre ellas estando a pocos metros o incluso a kilómetros de distancia. 

Es de justicia decir también, que al igual que estos sistemas permiten un gran número de usos, también tienen puntos débiles. Debemos poner de manifiesto que su programación a bajo nivel puede resultar bastante complicada al principio y en muchas ocasiones se necesitan placas específicamente creadas para algunos periféricos concretos, debido a su voltaje, número de pines o funcionamiento. Esto genera que cada vez que se hace un proyecto, dependiendo de la dificultad y dimensiones de este, se deba crear un sistema específico. Además por lo general disponen de poca memoria y por lo tanto requieren que las librerías y el propio código y variables utilizadas no sean demasiado extensos, ni requieran del almacenamiento de muchos datos.


\section{Líneas de trabajo futuras}\label{sec:LTF}

Este proyecto, al tratarse de un sistema embebido, puede ser continuado en varios puntos. Las opciones son prácticamente infinitas y solo se debe imaginar un objetivo para poder continuar. Algunos puntos importantes sobre los que se podría trabajar serian:

\begin{description}
\item La integración de la conexión vía wifi con el módulo ESP8266 o semejantes. De este modo podríamos replicar el mismo proyecto u otro diferente sin depender de la utilización de cables de red. 
\item Continuando con más conexiones, también podríamos utilizar la conexión bluetooth tanto para conectar la placa o bien a un móvil o a un ordenador de modo que pudiéramos utilizar un hardware que siempre tenemos a mano para gestionar los parámetros necesarios.
\item Otra línea de futuro sería implementar algún sistema de domótica, pudiendo llegar al punto de utilizarlo en tu propia casa. Algunos ejemplos serian, continuando con la idea del sensor de temperatura conectar o bien por infrarrojos o por bluetooth un SE al aire acondicionado y que este variara la temperatura automáticamente, incluso la implementación de un sistema de reconocimiento de voz.
\item Estas placas mediante protocolos IoT y servicios como por ejemplo Azure IoT Hub permiten conectar, supervisar y administrarlos. Además se pueden conectar a la nube para subir y descargarse datos de manera que pueden crear informes sobre el funcionamiento de un sensor.
\item Relacionado con la seguridad se podría tratar de cifrar todas las comunicaciones que haya entre dispositivos mediante bluetooth o Internet. En el caso de la conexión a Internet la propia pila que se ha utilizado en este proyecto, lwIP, dispone de un tipo cifrado conocido como \extranjerismo{Transport Layer Security} (TLS) de manera que las comunicaciones entre placas estuvieran cifradas.
\item En otros ámbitos con estas placas se podría realizar proyectos para realizar una acción, como por ejemplo abrir una puerta, mediante reconocimiento biométrico, bien facial o bien dactilar.
\end{description}


% Genera la bibliografía.
\bibliographystyle{plain}
\bibliography{bibliografia}

% Añade el aviso de la licencia.
\clearpage

\mbox{}
\vfill

\begin{figure}[!h]
  \centering
  \includegraphics[width=0.2\textwidth]{ccbyncsa}
\end{figure}

\begin{center}
  Este obra está bajo una
  \href{https://creativecommons.org/licenses/by-nc-sa/4.0/}
    {licencia de Creative Commons Reconocimiento-NoComercial-CompartirIgual 4.0
    Internacional}.
\end{center}

\end{document}
