\capitulo{2}{Objetivos del proyecto}

En este apartado de la memoria se detallan de forma concisa los objetivos de la realización de este proyecto. Por un lado tenemos los objetivos generales de este proyecto, por otro los objetivos técnicos y por último los objetivos que yo mismo me marco y por lo que realizo este TFG.

\section{Objetivos generales}\label{sec:OGenerales}
\begin{itemize}	
		\item Configuración de una red local LAN basada en ethernet de sistemas empotrados que sean capaces de conectarse y transmitir información entre ellos mediante los protocolos TCP/IP. Para ello se creará una relación punto a punto con sistemas embebidos.
		\item Uso de sistemas embebidos y exploración de todo su ecosistema, sensores, tipos de placas, configuración de periféricos usando otros protocolos, configuración de pines y todas sus funcionalidades.
		\item Demostración de las posibles utilidades que se le pueden dar a estos sistemas en distintos entornos mediante la utilización de distintos periféricos.
		\item Demostración de la correcta ejecución y uso de la red de sistemas empotrados.
\end{itemize}
	
\section{Objetivos técnicos}\label{sec:OTecnicos}
\begin{itemize}
	\item Programación de software en sistemas empotrados. Comprender el funcionamiento de este tipo de sistemas y la gestión de los procesos mediante sistemas operativos en tiempo real (RTOS).
	\item Configuración y programación de una comunicación I2C para la representación de caracteres por la pantalla LCD.
	\item Configuración y programación de una comunicación UART para el envío de comandos a los motores y conseguir así su correcto funcionamiento
	\item Manejo y configuración de conversores analógico-digitales (ADC).
	\item Programación de aplicaciones de red en un sistema empotrado mediante la implementación del modelo TCP/IP. 
	\item Comprender el funcionamiento de otros protocolos de red como DHCP o la configuración de un router y un switch.
\end{itemize}
			
\section{Objetivos personales}\label{sec:OPersonales}
\begin{itemize}
	\item Comprender el funcionamiento de los sistemas embebidos y sus utilidades en la vida real.
	\item Conocer el funcionamiento de envío y recepción de paquetes del protocolo TCP/IP.
	\item Aumentar mi conocimiento en el ámbito de redes. Establecimiento de una conexión, envío de paquetes, etc.
	\item Usar en un proyecto real las técnicas adquiridas durante la carrera en ámbitos como redes, organización de proyectos y programación.
	\item Utilizar la metodología Scrum vista en diferentes asignaturas del Grado.
	\item Emplear el sistema de control de versiones distribuido Git a través de la plataforma de desarrollo GitHub.
	\item Aprender a utilizar la herramienta LATEX para el desarrollo de documentación profesional.
\end{itemize}		

		