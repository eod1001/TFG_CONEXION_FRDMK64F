\capitulo{3}{Conceptos teóricos}

En esta sección se detallaran los conceptos teóricos necesarios para el desarrollo del proyecto. 

Algunos conceptos teóricos de \LaTeX \footnote{Créditos a los proyectos de Álvaro López Cantero: Configurador de Presupuestos y Roberto Izquierdo Amo: PLQuiz}.

\section{Sistemas Embebidos}

Los sistemas embebidos o empotrados son herramientas de computación programadas con una o varias funcionalidades concretas. 
Las grandes ventajas de estos sistemas son que trabajan de forma autónoma, ininterrumpida y sin necesidad de mantenimiento. Estas características hacen que su uso sea muy interesante para el sector industrial.[oasys]

\subsection{Hardware}

En los sistemas embeebidos. practicamente todos los componentes estan integrados en la placa base. Este hardware tambien provee la opcion de conectar perifericos mediante pines o entradas especificas para un periferico en cooncreto.[oasys] 

\subsubsection{Hardware del Proyecto}

//////////////////////
k64f y caracteristicas
esp8266

\subsection{Software}

El software embebido o empotrado reside en memoria de sólo lectura y se utiliza para controlar productos y sistemas de los mercados industriales y de consumo [redlyc].

\subsubsection{Software del Proyecto}

/////////////////// No se yo si poner esto
mcuexpresso
drivers, middleware, SO, etc
jlink


\section{Tecnologias de Conexion para SE}

Existen varias formas de comunicarse con los SE, tanto con componentes externos como entre dos o mas microcontroladores. Por lo general, se pueden conectar a los SE periféricos que lo doten de conexiones infrarrojas bluetooth wifi, entre otros. De esta manera conseguimos que podamos enviar y recibir información a estos sistemas. En este apartado vamos a hablar sobre las conexiones de red wifi y ethernet que son las que hemos usado en este proyecto.

PROTOCOLO UTILIZADO TCP/IP + CAPAS

\subsection{WIFI}



\subsection{Ethernet}

\section{Conexion Serial, I2C}



\section{Referencias}

Las referencias se incluyen en el texto usando cite \cite{wiki:latex}. Para citar webs, artículos o libros \cite{koza92}.
%- https://oasys-sw.com/sistemas-embebidos-industria/
%- https://www.redalyc.org/pdf/944/94415759009.pdf
- 

\subsubsection{NOTAS}

En oasys tienes mas informacion sobre el ambito industrial. beneficios y vulnerabilidades

\section{Imágenes}

Se pueden incluir imágenes con los comandos standard de \LaTeX, pero esta plantilla dispone de comandos propios como por ejemplo el siguiente:

\imagen{escudoInfor}{Autómata para una expresión vacía}



\section{Listas de items}

Existen tres posibilidades:

\begin{itemize}
	\item primer item.
	\item segundo item.
\end{itemize}

\begin{enumerate}
	\item primer item.
	\item segundo item.
\end{enumerate}

\begin{description}
	\item[Primer item] más información sobre el primer item.
	\item[Segundo item] más información sobre el segundo item.
\end{description}
	
\begin{itemize}
\item 
\end{itemize}

\section{Tablas}

Igualmente se pueden usar los comandos específicos de \LaTeX o bien usar alguno de los comandos de la plantilla.

\tablaSmall{Herramientas y tecnologías utilizadas en cada parte del proyecto}{l c c c c}{herramientasportipodeuso}
{ \multicolumn{1}{l}{Herramientas} & App AngularJS & API REST & BD & Memoria \\}{ 
HTML5 & X & & &\\
CSS3 & X & & &\\
BOOTSTRAP & X & & &\\
JavaScript & X & & &\\
AngularJS & X & & &\\
Bower & X & & &\\
PHP & & X & &\\
Karma + Jasmine & X & & &\\
Slim framework & & X & &\\
Idiorm & & X & &\\
Composer & & X & &\\
JSON & X & X & &\\
PhpStorm & X & X & &\\
MySQL & & & X &\\
PhpMyAdmin & & & X &\\
Git + BitBucket & X & X & X & X\\
Mik\TeX{} & & & & X\\
\TeX{}Maker & & & & X\\
Astah & & & & X\\
Balsamiq Mockups & X & & &\\
VersionOne & X & X & X & X\\
} 
