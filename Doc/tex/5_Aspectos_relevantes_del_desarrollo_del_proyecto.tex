\capitulo{5}{Aspectos relevantes del desarrollo del proyecto}

A medida que avanzaba con el proyecto han ido surgiendo algunos hitos importantes que voy a remarcar en este capitulo.

\section{IDE: Kinetis vs MCUXpresso}\label{sec:ARKinetisvsMCUX}
Como ya vimos en el apartado de herramientas se han utilizado dos \extranjerismo{IDEs}, ambos dos creados por la misma empresa, NXP. En un primer momento se comenzó utilizando Kinetis Design Studio puesto que mis tutores me facilitaron algunas practicas realizadas con ese programa de cara a famirializarme con ello. Tras la realización de las practicas enseguida me di cuenta de que estaba bastante anticuado y que las nuevas versiones de \extranjerismo{drivers} y \extranjerismo{middelware} no funcionaban en él, por lo que decidí migrar hacia MCUXpresso. Este cambio hizo que tuviera que volver a famirializarme con el nuevo IDE y la manera de programar el hardware a bajo nivel. Aunque KDS estuviera anticuado en la parte de programacion hardware, es decir en dotar a los pines de la placa de una funcionalidad y de sus respectivos relojes para el correcto funcionamiento, algo que MCUXpresso no tiene como tal y dificulta su configuracion. A favor de la version nueva diré que incluye decenas de programas de ejemplo con los que poder 'cacharrear' y comprender su funcionamiento, ademas de un montón de \extranjerismo{drivers} que habilitan la instalación de un gran numero de sensores. También se puede observar en MCUXpresso como la interfaz es mas limpia y simple a la hora de interactuar con ella. Una vez comprendes como funciona la programacion de periféricos, relojes y pines, lo cual no es sencillo en un principio, programar un sistema embebido se convierte en algo sencillo.

\section{FRDM K64F vs Arduino}\label{sec:ARK64FvsArduino}

Este proyecto también podría haberse desarrollado con las placas arduino UNO. Las cuales son muy parecidas en cuanto a funcionamiento y periféricos que permiten utilizar a la FRDM-K64F. Ademas, utilizar Arduino hubiera sido mas sencillo debido a que los pines vienen ya configurados, al igual que los relojes y tan solo hubiera sido enchufar y programar las funciones de las tareas que quisiéramos. Arduino también cuenta con una comunidad mayor por lo que tendríamos librerías mas simples y avanzadas y mayor información en Internet de cara a resolver fallos. Entonces, ¿Por qué decidí utilizar las FRDM K64F?\\
La respuesta es simple, el objetivo de la realización de este trabajo no era simplemente desarrollar un laboratorio. El objetivo principal no es que los motores funcionen adecuadamente, o la pantalla, o el sensor de temperatura, etc. El objetivo principal era aprender sobre el funcionamiento de los sistemas embebidos. Es por ello que se decidieron utilizar estas placas, para poder aprender como funciona el microcontrolador y sus extensiones mediante los pines de la placa. Se ha pretendido desde el principio centrar los esfuerzos en comprender como se realiza la configuración de sus pines y relojes, etc. En un entorno real se deben conocer el funcionamiento de estos sistemas a bajo nivel puesto que cada proyecto desarrollado en el entorno laboral necesitara que el SE cumpla con unos requisitos específicos y no exceda de ellos para disminuir el costo y tamaño del sistema. En cada proyecto se desarrollan unas placas especificas para esos requisitos y es conveniente saber su funcionamiento interno para poder diseñarlas correctamente.
Por otro lado, en el caso de arduino probablemente no hubieramos podido utilizar ni FreeRTOS ni lwIP debido a que su potencia es menor a la de las placas K64F. Además con arduino tampoco hubieramos podido debuguear puesto que no incluye reloj en tiempo real.


\section{Ethernet vs Wifi}\label{sec:ETHvsWIFI}
Como ya vimos en el apartado comunicaciones por red, existen dos formas de comunicarse, una es por cable de red ethernet y la otra vía wifi utilizando una antena wifi. En este punto veremos las características para cada uno para poder entender por qué nos decidimos por hacerlo por cable en el trabajo.
Pese a que las redes wifi se han convertido en la forma mas común de conectarnos en nuestros hogares y lugares de ocio no son las mas apropiadas en todos los casos. Este tipo de conexión no es tan segura, ni estable como la conexión por cable. Veamos las ventajas e inconvenientes de cada una.

Usar cable de red ethernet es mucho más rápido que usar una conexión WiFi. Según un estudio realizado por ADSLZone \cite{AdslWifivsEthernet}, se pierde hasta el 65\% de tu conexión a Internet mediante wifi. Esto se debe a factores como la distancia, obstáculos o interferencias con otros dispositivos. Por otro lado, si vives en una comunidad de vecinos o estas rodeados de varias conexiones wifi también se puede producir una saturacion de los canales que utilizan estas redes.
En el caso del uso de cable no tienes ninguno de los inconvenientes anteriores sin embargo existe una gran desventaja y que resulta bastante mas engorroso tener que llevar un cable de red hasta cada equipo para que pueda conectarse a Internet. Ademas nos obliga a disponer de conexiones \extranjerismo{switch} y en algunos casos habrá que configurar estos \extranjerismo{switches} según nuestras necesidades. En el caso de este proyecto solo se iban a utilizar 3 placas que ademas se iban a conectar entre ellas por lo que aparentemente el uso de ethernet es mas sencillo.

Por todos estos motivos y puesto que la placa nos otorgaba la posibilidad de utilizar cable de red, decidí usar esta opción.


\section{RTOS}\label{sec:ARRTOS}
En el caso de los sistemas operativos en tiempo real encontramos algunas alternativas a FreeRTOS. Las características y conceptos mas importantes de FreeRTOS ya los vimos en el apartado \ref{sec:RTOS}. Como alternativa a este sistema operativo encontramos: 
\begin{description}
\item \extranjerismo{Embedded Operating System} (embOS), es un sistema operativo en tiempo real, desarrollado por la empresa SEGGER Microcontroller. Está diseñado para ser utilizado como base para el desarrollo de aplicaciones integradas en tiempo real para una amplia gama de microcontroladores. El funcionamiento es prácticamente el mismo.
\item MQX es otra opcion de sistema operativo en tiempo real propuesto por NXP. Es un SO que ofrece una API sencilla y una arquitectura modular que hace que este software sea escalable.
\end{description}
El motivo de haber elegido la utilización de FreeRTOS es su presencia en un mayor numero de proyectos que sus competidores. Esto hace que existan comunidades en Internet que nos brindan mayor información y soluciones para su correcta utilización. Ademas de que su propio manual ya nos ofrece los pasos a seguir, es realmente sencillo de utilizar, una vez has aprendido los conocimientos básicos de su funcionamiento.

\section{Metodologías Ágiles: SCRUM}\label{sec:ARMetodologiasAgiles}

Como alternativas a SCRUM teníamos dos metodologías ágiles muy utilizadas y quizás mas sencillas de implementar:

\begin{description}
\item[Extreme Programming XP]
Esta metodología es muy utilizada en pequeñas empresas durante sus comienzos y consolidación. Se basa en centrar sus esfuerzos en la comunicación entre clientes y empleados potenciando las relaciones personales mediante el trabajo en equipo y promoviendo la comunicación y la eliminación de tiempos muertos.
Sus principales fases son:
\begin{enumerate}
\item Diseño y planificación del proyecto con el cliente
\item Programación por parejas dentro del equipo de forma que ambos puedan intercambiar conocimientos consiguiendo mejores resultados
\item Realización continua de pruebas de código.
\end{enumerate}

\item[Kanban] 
La metodología Kanban se basa en el uso de un \extranjerismo{layout} con columnas, generalmente tres, en las que se muestran las tareas que quedan por hacer: "Pendientes", las que están en curso: "En proceso" y las que ya se terminaron: "Terminadas". Cada usuario o equipo tiene la posibilidad de aumentar el numero de columnas según les sea de mayor utilidad. D esta manera se tiene conocimiento sobre el estado del proyecto en tiempo real, mejorando la productividad y eficiencia del desarrollo del trabajo.
Las principales ventajas de esta metodología son:
\begin{enumerate}
\item Facilidad para la planificación de tareas
\item Mejora en el rendimiento de trabajo del equipo
\item Visión global de un solo vistazo
\item Los plazos de entregas son continuos
\end{enumerate}
\end{description}

Pese a estas dos alternativas se decidió usar SCRUM ya que es la metodología mas estudiada durante el grado. Ademas, el uso de la herramienta GitHub hace que sea fácil de usar y también consigue que la generación de código quede bien expuesta y organizada.

\section{Dificultades y Problemas}\label{sec:ARDificultades}
Durante el desarrollo de todo el proyecto he tenido algunos inconvenientes tanto personales como técnicos. Veamos algunos de ellos:
\begin{description}
\item Ya desde un principio perdí tiempo por el cambio de IDE y tener que familiarizarme de nuevo con las interfaces del programa. 
\item Por otro lado, la parte de configuración de pines, periféricos y relojes es algo complicada, sobretodo al principio del proyecto ya que durante el grado no se ha visto nada parecido. A la hora de buscar información sobre como realizar algún procedimiento, tanto en código, como en configuración de pines, no encontraba demasiada información. El uso de estas placas es algo especifico ya que están pensadas para usuarios con ciertos conocimientos en el ámbito de los SE. Al mismo tiempo va dirigido para personas que tampoco son expertos en la materia y no saben diseñar sus propias placas. Esto hace que se cierre mucho el nicho de personas a las que van dirigidos.
\item Relacionado con la organización y planificación fue complicado puesto que mi situación personal fue cambiando a lo largo del proyecto. Al principio de su realización no tenia demasiadas obligaciones, lo que me permitía tener un cierto control en cuanto a horas y horarios diarios. A mitad del desarrollo comencé a realizar las practicas curriculares y el tiempo libre disminuye complicando tener una estabilidad diaria para realizar este proyecto.
\item Volviendo al desarrollo, la adaptación del tipo de comunicaciones, sobretodo la uart, al envío de los comandos de los motores, fue algo compleja. Estas dificultades vinieron dadas en gran parte por los cambios de tipos.
\end{description}
Estos han sido los hitos que mas dificultades me han causado durante la realización del TFG.









