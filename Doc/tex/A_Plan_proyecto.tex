\apendice{Plan de Proyecto Software}

\section{Introducción}

En este apartado se describen los resultados obtenidos tras la realización del proyecto. 

Se trataran dos temas principales. El primero corresponde a la planificación temporal del trabajo, en la cual veremos las tareas realizadas y los tiempos requeridos para llevarlas a cabo. También veremos el tiempo total de la realización del trabajo de fin de grado. Puesto que se esta utilizando una metodología Scrum algunos tiempos pueden no ajustarse lo suficiente a lo esperado sobretodo los de los primeros sprint realizados ya que en estos se debe llevar un proceso de medición de fuerzas y dificultades del trabajo a realizar. 
el segundo tema del que se va a desarrollar es la realización de un análisis de la viabilidad de un proyecto de estas características en la vida real. Para ello nos centraremos en analizar las variables económicas y legales. En cuanto a este ultimo punto veremos los tipos de licencias que se pueden utilizar en este tipo de software


\section{Planificación temporal}

Como ya hemos comentado la planificación del proyecto se llevo a cabo siguiendo la metodología Scrum que esta basada entre otras cosas, en el uso de \extranjerismo{sprints} para racionar y organizar la carga de tareas. 

Según esta metodología estos \extranjerismo{sprints} deben durar entre 1 y 3 semanas. En este caso y puesto que mi situación personal ha sido distinta a lo largo de este proceso la duración de estos \extranjerismo{sprints} no ha sido siempre la misma. Esto se debe a que mis obligaciones han ido variando a medida que avanzaba el cuatrimestre. Al principio de este, durante los 3 primeros \extranjerismo{Sprint} y comienzos del cuarto disponía de bastante mas tiempo para su realización por lo que era capaz de meter mas horas al día, posteriormente empece a cursar las practicas curriculares con un total de seis horas diarias lo que dificulto seguir con el mismo ritmo de horas dedicadas al TFG. Si que se ha tratado de que los \extranjerismo{sprints} estuvieran formados por aproximadamente el mismo numero de horas. Estas horas están expuestas en el repositorio de Git Hub y cada tarea tiene asignado un \extranjerismo{tag}. Cada uno de estos \extranjerismo{Tags} es un numero que relaciona 1 a 1 la dificultad con las horas para su realización, es decir, una dificultad de ocho equivale a aproximadamente 8 horas de trabajo. Como es lógico cuanta mayor es la dificultad mas se rompe esta relación, puesto que al ser mas complicado siempre suele alargarse el tiempo de realización de la tarea. 
El proyecto se desarrollo en los siguientes Sprints:
\begin{description}
\item Sprint 1: Investigación y famirializacion (Software, Hardware y Herramientas)
\item Sprint 2: Envío y Recibo de paquetes TCP/IP
\item Sprint 3: Investigación y primeros pasos con periféricos (Comunicación uart, I2C)
\item Sprint 4: Investigación sobre Wifi. Uso de potenciometros para gestionar motores
\item Sprint 5: Periféricos (sensor de temperatura y pantalla) y uso real en una empresa
\item Sprint 6: Memoria, presentación y últimos Retoques
\end{description}
En los siguientes apartados obtendremos mas información sobre como fue el desarrollo de cada uno de estos Sprints.


\subsection{Sprint 1}
Este Sprint esta compuesto por 10 tareas que en total suman 43 puntos. Duró aproximadamente dos semanas. El comienzo fue complicado por la gran cantidad de conceptos y funcionamientos del hardware que debí investigar, ademas de la dificultad de famirializarse con dos IDEs diferentes. Las acciones principales fueron:
\begin{description}
\item Elección, descarga y famirializacion del entorno de desarrollo
\item Completar las practicas propuestas por los Cotutores
\item Descarga de otros programas para documentación, organización y comunicación
\item Primeros pasos con lwIP
\item Primeros pasos con RTOS
\end{description}

\imagen{Sprint1}{Tareas planificadas para el Sprint1.

Como podemos ver en la figura anterior, este primer Sprint se basó en la creación del entorno de trabajo y primeros pasos con las herramientas que iba a usar. Estos primeros pasos incluían ya la utilización de botones, leds y configuración de algún periférico.

\subsection{Sprint 2}
El sprint 2 recoge 9 tareas que suman 35 puntos de dificultad. Tuvo una duración de dos semanas y media. Este sprint estuvo enfocado a comenzar con la investigación del funcionamiento de lwIP. Como funcionaba y para que servia fueron los dos puntos mas importantes. Tras entender estos dos puntos busque información sobre como implementarlo y comencé por la obtención de una IP según la mac de la placa y la creación de una tarea que pudiera recibir paquetes con información a través de este servicio. Para enviar los primeros paquetes utilice Packet Sender.
\begin{description}
\item Investigación Protocolo TCP/IP y su implementacion con lwIP
\item Primeras partes del software
\item Pruebas con la herramienta Packet Sender
\end{description}

\imagen{Sprint2}{Tareas planificadas para el Sprint2.

Otra parte a destacar fue el comienzo de la memoria con la herramienta Latex. Una herramienta totalmente nueva para mi.

\subsection{Sprint 3}
En este Sprint se buscó conseguir la comunicación de tres placas mediante el protocolo TCP/IP. Recordemos que ya disponíamos de conexión en red y de un \extranjerismo{"Listener"} que recibía paquetes de datos. El objetivo ahora era crear ademas un \extranjerismo{"Writer"} capaz de enviarlos. Ademas también se decidió la manera en que se transmitiría la información y seria en forma  de comandos, por lo tanto también se implemento una especie de filtro que gestionaba el tipo de comando recibido en caso de existir. Por otro lado tambien se comenzo a investigar e implementar las comunicaciones con los perifericos, concretamente con los motores y la panatalla. En este caso el sprint esta constituido por 8 tareas que suman 36 puntos. Tareas principales:
\begin{description}
\item Creacion del \extranjerismo{"Writer"}
\item Creacion del filtro de comandos
\item Primeros pasos con uart y comunicacion con los motores
\item Primeros pasos con I2C y la pantalla LCD.
\end{description}

\imagen{Sprint3}{Tareas planificadas para el Sprint3.

Este sprint tuvo una duración de dos semanas.

\subsection{Sprint 4}
En este Sprint se terminan la comunicación de los motores y la de la pantalla quedando operativas a falta de algunas mejoras. Ademas se produce uno de los hitos mas importantes del proyecto que es la investigación y abandono del procedimiento para la instalación del modulo wifi por falta de tiempo. En este momento estoy comenzando las practicas y decido dejar aparatada la parte de conexión inalambrica para centrarme en la obtención de mas funcionalidades. Esto vienen dado tanto por la falta de tiempo que hablábamos anteriormente y también porque desde la empresa en la que curso las practicas me ofrecen la posibilidad de hacer algo que pudiera usarse en la fabrica.
Debido a esto también me pongo a investigar sobre la posibilidad de introducir potenciometros y termómetros para poder medir la temperatura de su CPD. El sprint suma una dificultad de 42 puntos en 8 tareas se realizo en una semana y media. Tareas principales:
\begin{description}
\item Investigación sobre la implementacion de un modulo WIFI, ESP8266.
\item Mejoras del código referente a la comunicación con los motores y la pantalla LCD.
\item Investigación e implementacion de la lectura de los potenciometros.
\end{description}

\imagen{Sprint4}{Tareas planificadas para el Sprint4.

\subsection{Sprint 5}
Este sprint contiene 9 tareas llegando a un total de 32 puntos de dificultad. Es en este sprint donde se implementa la lectura del sensor de temperatura. Ademas se centran muchos esfuerzos en tratar de conseguir que todas las tareas funcionen adecuadamente. Se realizan multitud de pruebas y se trata de conseguir que la interacción con el usuario sea simple e intuitiva. Las tareas principales se resumen en:
\begin{description}
\item Mejora de usabilidad por el usuario
\item Implementacion del sensor de temperatura
\item Mejora del código y funcionalidades de los periféricos.
\end{description}

\imagen{Sprint5}{Tareas planificadas para el Sprint5.
La duracion del sprint fue de una semana y media.
\subsection{Sprint 6}
Este Sprint se centra en dar los últimos retoques y completar la documentación a presentar para superar el TFG. Es el sprint mas largo con 76 puntos de dificultad. En este caso se deberían haber separado en dos Sprints pero puesto que todas las tareas se trataban sobre la documentación se decidió dejar en solo uno. La duración de este sprint fue de dos semanas y media.
\begin{description}
\item Últimos retoques al código
\item Generar la documentación de todo el trabajo.
\end{description}

\imagen{Sprint6}{Tareas planificadas para el Sprint6.






\section{Estudio de viabilidad}

\subsection{Viabilidad económica}
En este apartado se expone la viabilidad económica del desarrollo de este proyecto. Se tendrán en cuenta los gastos previstos durante el desarrollo y los beneficios si los hubiera. Para calcularlo supondremos que el trabajo se hubiera desarrollado en el entorno real de una empresa. 

\subparagraph{Coste de personal}
Para el desarrollo del producto final vamos a suponer que se hubiera encargado una sola persona, durante un tiempo aproximado de 90 días laborales y 25 horas semanales. Supongamos que este trabajador junior tenga un salario bruto de 1800 euros. 
Para conocer el salario neto lo haremos con la siguiente formula:
\begin{equation} \label{eq:salario}
  salario\ bruto - IRPF - SS = salario\ neto
\end{equation}
Por lo que el trabajador recibiría el siguiente salario:
\begin{equation} \label{eq:salario}
  1800€ - 29\% - 6,35\% = 1163,7€
\end{equation}
	
\subparagraph{Costes pertenecientes a la SS}
Sobre el salario bruto se producen retenciones y pagos a la Seguridad Social. De estos impuestos algunos los paga la empresa y otros el trabajador. En este apartado se van a calcular los importes de estas retenciones, para ello tomaremos como referencia la tabla de bases de cotización de la Seguridad Social. En esta tabla deberemos buscar el grupo que nos corresponda, en este caso "Ingenieros y Licenciados". Personal de alta dirección no incluido en el articulo 1.3. c) del Estatuto de lo Trabajadores". Este grupo tiene unas bases mínimas de 1466,40€/mes. Y unas máximas de 4070,10€/mes.

\tablaSmallSinColores{Costes pertenecientes a la SS}{l l l}{costes-ss}
{\multicolumn{1}{l}{Concepto} & Empresa & Trabajador\\}
{
  Contingencias comunes & 23,60\% & 4,70\%\\
  Desempleo             &  5,50\% & 1,55\%\\
  FOGASA                &  0,20\% & 0,00\%\\
  Formación             &  0,60\% & 0,10\%\\
  \textbf{Total}        & \textbf{29,9\%} & \textbf{6,35\%}\\
}

Como podemos observar en la figura anterior los costes que la empresa debe asumir son el 23,9\% del salario del trabajador y el trabajador sufriría una retención del 6,35\% 

\subparagraph{Coste total de personal}
Para hallar el coste total del desarrollo del proyecto calcularemos la siguiente formula.
\begin{equation} \label{eq:salario}
  (salario\ mensual + retenciones\ ss) \times n^{o}\ meses = coste\ total
\end{equation}

De modo que el coste total se obtiene así:
\begin{equation} 
  (1800€ + *****) * 3 = ////€
\end{equation}

La tabla siguiente tabla recoge diferentes costes asociados al personal y 
su coste total.

\tablaSmallSinColores{Coste total de personal}{l l}{personal}
{\multicolumn{1}{l}{Concepto} & Coste\\}
{
  Salarios          & \EUR{1800}\\
  Seguridad Social  & \EUR{////}\\
  Meses             & 3 meses\\
  \textbf{Total}    & \textbf{\EUR{////}}\\
}


\subparagraph{Costes del hardaware}
Como ya vimos en la memoria para el desarrollo de este proyecto se han necesitado varios dispositivos. Si bien es cierto que en la Ley del Impuesto sobre Sociedades se indica un maximo de 8 años para amortizar los equipos para procesos de información, se considera la que renovacion de hardware se produce en un periodo mas corto de 4 años. Por lo tanto se hallara un resultado teniendo en cuenta una amortización de 4 años.

El coste mensual de un dispositivo se calcula así:
\begin{equation} \label{eq:coste-hw}
  coste\ del\ dispositivo\ /\ periodo\ de\ amortización\ (en\ meses)
\end{equation}

Por lo tanto, el coste de un dispositivo es el siguiente:
\begin{equation} \label{eq:coste-amor-hw}
  coste\ mensual\ amortizado\ \times n^{o}\ meses\
\end{equation}

En el desarrollo solo se va a necesitar una estación de trabajo, así pues, el
coste total del \extranjerismo{hardware} es el mostrado en la tabla a
continuación.

\tablaSmallSinColores{Coste del \extranjerismo{hardware}}{l l l}{coste-hw}
{\multicolumn{1}{l}{\extranjerismo{Hardware}} & Coste & Coste amortizado\\}
{
  Estación de trabajo & \EUR{1250} & \EUR{104,16}\\
  \textbf{Total}      & \EUR{1250} & \textbf{\EUR{104,16}}\\
}




\subparagraph{Costes del software}
////COPIADO
El desarrollo necesita de varios programas y aplicaciones.  
El desarrollo necesita de varios programas y aplicaciones. De igual manera
que el hardware, la Ley del Impuesto sobre Sociedades [4] indica el máximo de
años para realizar la amortización. En los sistemas y programas informáticos
es de 6 años. Pero coincidiendo con el periodo de renovación del hardware,
la amortización se calcula en un periodo menor de 4 años.
Las fórmulas para calcular las amortizaciones son equivalentes a las
usadas en el harware A.5 y A.6.
Otra parte de software se emplea bajo suscripción, en los cálculos solo
se computa el coste de los meses que se usó dicha suscripción.
Hay que tener en cuenta, que gran parte del software utilizado se publica
bajo licencias que permiten su uso sin coste y que gracias a esto se reducen
significativamente los costes del software. En consecuencia solo se calculan
los costes del software que no es gratuito.

\tablaSmallSinColores{Coste del \extranjerismo{software}}{l l l}{coste-sw}
{\multicolumn{1}{l}
{\extranjerismo{Software}}              & Coste        & Coste amortizado\\}
{
  Windows 10 Pro\cite{webpage:win10pro} & \EUR{259}    & \EUR{21,58}\\
  Office 365\cite{webpage:office365}    & \EUR{126}    & \EUR{31,5}\\
  \textbf{Total}                        & \EUR{744,88} & \textbf{\EUR{143,05}}\\
}


\subparagraph{Coste del sistema empotrado}

\tablaSmallSinColores{Coste del Sistema Empotrado}{c c c c}{coste-SE}
{\multicolumn{1}{l}{\extranjerismo{Hardware}} & Coste & Unidades & Coste amortizado\\}
{
  Placas FRDM-K63F & \EUR{} &  & \EUR{}\\
  Shield Arduino & \EUR{} &  & \EUR{}\\
  Pantalla LCD & \EUR{} &  & \EUR{}\\
  Motores & \EUR{} &  & \EUR{}\\
  Switch & \EUR{} &  & \EUR{}\\
  \textbf{Total} & \EUR{} &  & \textbf{\EUR{}}\\
}


\subparagraph{Coste total del proyecto}
Sumando todos los costes anteriores se puede calcular el coste total de todo
el proyecto.

\tablaSmallSinColores{Coste total del proyecto}{l l}{coste-total}
{\multicolumn{1}{l}
{Tipo de coste}            & Coste        \\}
{ 
  Personal                 & \EUR{8573,4} \\
  \extranjerismo{Hardware} & \EUR{104,16} \\
  \extranjerismo{Software} & \EUR{143,05} \\
  Componentes del SE       & \EUR{82,53}  \\
  \textbf{Total}           & \textbf{\EUR{8903,14}} \\
}

\subparagraph{Beneficios del proyecto}
El código fuente del sistema empotrado (SE) y de la aplicación web se encuentran
disponibles abiertamente. Además, el SE tampoco tiene definida una función
claramente comercial. Por ello, no se considera que el proyecto
pueda tener un beneficio económico directo.

Esto tampoco quiere decir que los gastos se hagan a fondo perdido. Más bien,
se pueden considerar a los gastos como una inversión en investigación, formación
y adquisición de competencias que permitirán en el futuro crear nuevos SE que
rentabilicen todos los costes.


\subsection{Viabilidad legal}
Al desarrollar un \extranjerismo{software} hay que tener en consideración
las implicaciones legales que se presentan al usar \extranjerismo{software}
de terceros.

Las licencias sirven como instrumento para establecer los términos y condiciones
en los que se pueden utilizar el \extranjerismo{software} licenciado. Para
poder definir que licencias usar en el proyecto es necesario conocer las
licencias utilizadas y la restricciones que pueden imponer.

\subsubsection{Licencias utilizadas en el desarrollo del SE}
Para desarrollar el \extranjerismo{software} de la placa se ha utilizado
de código fuente de terceros que se describe a continuación.
\begin{itemize}
  \item El código generado por MCUXpresso se proporciona bajo la licencia
  ``The 3-Clause BSD License'' (BSD-3), también conococida como la
  ``New BSD License''o ``Modified BSD License''.
  \item Otra parte del código generado por MCUXpress se licencia con 
  ``The Clear BSD License''. La licencia es similar a BSD-3, pero indica
  expresamente que no se conceden derechos sobre patentes.
  \item El código perteneciente a FreeRTOS es propiedad de Amazon.com, Inc.
  Se permite el uso, copia, modificación, publicación, distribución, volver a
  licenciar y comercializar, manteniendo siempre el aviso de derechos de autor.
  \item El código de lwIP se entrega con la licencia BSD-3.
  \item Parte del código relacionado con el procesador de la placa pertenece a
  ARM Limited. El código se proporciona con la licencia Apache 2.0.
\end{itemize}

Así pues, se encuentran licencias como BSD-3 o Apache 2.0 que permiten el uso,
copia, modificación y redistribución del código.



\subparagraph{Licencias para el SE}
Teniendo en cuanta que el código fuente del \extranjerismo{software} del SE
se pretende que sea abierto, se va a utilizar una licencia en línea con las
mostradas anteriormente.

En concreto se va a emplear la licencia Apache 2.0. Sus características 
principales se pueden ver resumidas en la siguiente tabla.

\tablaSmallSinColores{Licencia Apache 2.0}{l l l l}{apache}
{\multicolumn{1}{l}
{Permisos}        & Condiciones                & Limitaciones    \\}
{ 
  Uso comercial   & Aviso de licencia          & Uso de marcas registradas \\
  Modificación    & y derechos de autor        & Responsabilidad           \\
  Distribución    & Declaración de los cambios & Garantía                  \\ 
  Uso en patentes \\
  Uso privado     \\
}
\subparagraph{Utilización de la licencia Apache}
Para poner en conocimiento el uso de la licencia Apache
\cite{webpage:apache2-apply} se ubica en el directorio raíz del proyecto un
archivo de texto con el nombre ``LICENSE'' que incluye los términos de la
licencia. El archivo de texto se puede obtener desde el sitio web de Apache
\cite{webpage:apache2-license}.

En caso de ser necesario, se incluye un fichero llamado ``NOTICE'' con
información adicional en el mismo directorio de la licencia.

Por último, el texto a continuación es incluido en todos los ficheros 
fuente. Se tiene que añadir como comentario en el comienzo del fichero,
sustituyendo el texto entre corchetes por el nombre del autor y por la fecha
que correspondan.

\begin{quotation}
  Copyright [yyyy] [name of copyright owner] \bigskip

  Licensed under the Apache License, Version 2.0 (the "License");
  you may not use this file except in compliance with the License.
  You may obtain a copy of the License at \bigskip
  
  \quad http://www.apache.org/licenses/LICENSE-2.0 \bigskip

  Unless required by applicable law or agreed to in writing, software
  distributed under the License is distributed on an ``AS IS'' BASIS,
  WITHOUT WARRANTIES OR CONDITIONS OF ANY KIND, either express or implied.
  See the License for the specific language governing permissions and
  limitations under the License.
\end{quotation}


\subsubsection{Licencia para la documentación}
Para el licenciamiento de la documentación se opta por otro tipo de licencia
más adecuada al tipo de obra. La licencia escogida es
Creative Commons Atribución-NoComercial-CompartirIgual 4.0 Internacional
(CC BY-NC-SA 4.0) \cite{webpage:cc}.

Esta licencia permite que la obra sea compartida libremente. Puede ser copiada
y redistribuida en cualquier medio o formato. También se puede reeditar,
transformar y crear obras derivadas a partir de la obra original. Como condición
se requiere la atribución de la autoría original y que nuevas publicaciones
se realicen de forma gratuita y con el mismo licenciamiento que la actual.

\tablaSmallSinColores{Licencia CC BY-NC-SA 4.0}{l l l l}{ccbyncsa4}
{\multicolumn{1}{l}
{Permisos}      & Condiciones                & Limitaciones    \\}
{ 
  Distribución  & Crédito al autor           & Responsabilidad           \\  
  Modificación  & Aviso de licencia          & Uso de patentes           \\
  Uso privado   & y derechos de autor        & Uso de marcas registradas \\
                & Declaración de los cambios & Garantía                  \\ 
                & No comercial               &                           \\ 
                & Mismo licenciamiento       &                           \\ 
}

El uso de la licencia se indica con una imagen y un breve texto al final de cada
documento.
