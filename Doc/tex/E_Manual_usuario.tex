\apendice{Documentación de usuario}

\section{Introducción}
Esta sección recoge las instrucciones y conocimientos tecnicos que un usuario debe conocer para poder utilizar el laboratorio presentado.
Veremos como instalar el software en nuestra placa y como controlar la placa maestro. Se explicaran que funciones realizan cada boton y cuales son los usos que tiene el laboratorio.

\section{Requisitos de usuarios}

Para poder utilizar el programa lo primero es disponer de todas las piezas que componen el laboratorio.
Deberemos comprobar que todos los componentes están correctamente conectados entre si. Para ello deberemos:
\begin{enumerate}
\item Disponer de tres cables de red y conectar cada uno de ellos a una placa y a un switch.
\item Comprobar que los motores están correctamente conectados a la placa controladora y que las placas FRDM k64F también están conectadas a la placa controladora de los motores. Es importante cerciorarnos de que todos los pines están debidamente conectados a su posición  correspondiente
\item Tendremos que asegurarnos que todos los componentes están conectados a una toma de corriente y reciben energía, por lo general si están recibiendo corriente tendrán algún led encendido.
\item También deberemos comprobar que el switch cumple con su función mirando si todos los puertos donde se conectan los cables de red están recibiendo y enviando datos
\item Se deberá conectar la placa de expansion a la placa maestro, en este caso es muy sencillo porque solo existe una forma de conectarla para que todos los pines queden conectados.
\item Por ultimo deberemos conectar los pines de la pantalla lcd a una de las placas maestro para poder ver de manera adecuada la información de los comandos recibidos.
\end{enumerate}

Tras tener configurado el entorno el siguiente paso sera cargar el programa correspondiente a cada micrcontrolador. Recordemos que disponemos de tres placas y tres softwares con algunas diferencias entre ellos. En el apéndice anterior ya vimos como podíamos realizar este procedimiento.
Una vez se tienen los códigos cargados en las placas deberemos esperar a que los tres microcontroladores adquieran su respectiva dirección ip y podremos comenzar a enviar comandos desde la placa maestro como veremos en el siguiente apartado.



\section{Manual del usuario}
En el apéndice /// pudimos ver en los requisitos funcionales todos los usos de la placa y sus correspondientes pasos. En este apartado veremos de una forma mas gráfica las utilidades de la misma.

En primer lugar vamos a visualizar y nombrar los botones de la placa maestro.

%\imagen{placaMaestroBotones}{Botones de la placa maestro}

\begin{description}
\item Para configurar la velocidad del motor A deberemos mover el potenciometro de la placa según la velocidad (sentido del giro) que le queramos. Si lo giramos hacia la derecha completamente el motor recibirá el valor 255 y girará hacia la derecha, si por el contrario lo giramos hacia la izquierda recibirá el valor de 0 y girara en el sentido opuesto. En caso de que dejemos el potenciometro en un rango medio de su giro, el motor recibirá el dígito 128 y dejara el motor parado. 
\item Lo mismo pasa con el potenciometro 2 y el botón 2. En ambos casos, al pulsar el boton 1 o dos, recibiremos un \extranjerismo{feedback} en la placa esclavo mediante un mensaje en la pantalla que nos mostrara si se recibió el comando o no.
\item En caso de que queramos parar ambos motores utilizaremos el botón 3 a modo de parada de emergencia. 
El botón 4 envía a la placa maestro la temperatura ambiente. Esta temperatura se muestra en ºC por la pantalla del esclavo. 
\item Por ultimo los botones 5 y 6 muestran la velocidad de los motores A y B respectivamente, esta información se mostrara por la pantalla de la placa esclavo.
Es importante conocer que en muchos casos las primeras lecturas tanto de los potenciometros como del sensor de temperatura no le ha dado tiempo a leer la nueva posición y se debe repetir la acción para mandar el comando.
\end{description}

////////////////FOTOS


