\capitulo{7}{Conclusiones y Líneas de trabajo futuras}
{\label{ch:conclusiones-lineas}}

\subsection{Conclusiones}{\label{sec:conclusiones}}
Tras finalizar el proyecto han surgido varias conclusiones relativas al mismo.

\begin{itemize}
  \item Al concebir el proyecto se establecían una serie de objetivos generales
  \ref{sec:obj_generales}, técnicos \ref{sec:obj_tecnicos} y personales
  \ref{sec:obj_personales} que se han completado satisfactoriamente.
  Se ha logrado crear un SE \ref{sec:se} conectado en red gracias al uso de
  la familia de protocolos TCP/IP \ref{sec:tcpip}. De manera más específica,
  se ha utilizado una placa de desarrollo K64F \ref{sec:k64f} junto con lwIP,
  una implementación de la pila de protocolos TCP/IP \ref{sec:otros} para
  conectar en red dicha placa.
  
  \item Para su realización, se han puesto en práctica los conocimientos 
  aprendidos en el trascurso del Máster Universitario en Ingeniería Informática.
  Conocimientos en sistemas empotrados, arquitecturas y servicios de internet,
  gestión y dirección de proyectos, etc.

  \item A la vez se ha aprovechado esta oportunidad para adquirir nuevos
  conocimientos. Como se indica en los aspectos relevantes del proyecto
  \ref{sec:aprendizaje}, uno de los principios esenciales del proyecto ha sido
  aprender nuevas herramientas, técnicas y metodologías; y revisar aquellas ya
  conocidas.

  \item Ha resultado un descubrimiento conocer la cantidad de contextos en los
  que puede aparecer un SE. Era conocida su presencia en entornos con fuerte
  presencia de la tecnología de la información, p. ej., electrónica de consumo,
  automatización industrial, control de viviendas y edificios, interfaces
  hombre-máquina;pero su aparición en otros ámbitos como el sector de la salud
  donde su uso se extiende desde camas eléctricas \ref{sec:cama} hasta bombas
  de insulina ha sido sorprendente.

  \item Desarrollar un SE se convierte en un desafío en comparación con el
  desarrollo para sistemas convencionales. Las restricciones del 
  \extranjerismo{hardware} se traducen en una necesidad constante de ahorrar
  espacio en memoria, limita la cantidad y el tamaño de librerías a usar,
  insta a usar implementaciones livianas como en el caso de lwIP y dado el
  caso que lo requiera, apremia el uso de soluciones que reduzcan el consumo
  de energía por parte del sistema.

  \item Aunque la pila de protocolos TCP/IP no sean los únicos tipos de 
  protocolos usados por los SE su empleo ha facilitado el desarrollo con la
  placa y su posterior uso. Con inicializar la pila de protocolos era suficiente
  para poder enviar un \extranjerismo{ping} y ver como la  placa respondía 
  satisfactoriamente.

  \item Los mecanismos de depuración incluidos en la placa de desarrollo y en el
  IDE \ref{sec:ide} han demostrado su valía. En numerosas ocasiones mostrar
  en la consola de depuración un mensaje con el estado de la ejecución ha sido
  de gran utilidad para corregir errores. Sin ir más lejos, hasta la
  incorporación de la pantalla LCD, un mensaje de depuración era la única manera
  de saber la dirección IP asignada a la placa.

  \item Por lo que se refiere al desarrollo de la aplicación web, el aprendizaje
  y empleo de CSS ha servido para mostrar que se puede hacer en términos de
  presentación y diseño visual; y como se puede hacer. El desarrollo también
  ha servido para revisar las tecnologías usadas: Java EE, Enterprise Beans,
  Java Server Faces, GlassFish, Maven, etc.
 
  \item Con el uso de Scrum se consiguió repartir las tareas del desarrollo
  de forma práctica y efectiva. Siguiendo la planificación general, el
  desarrollo se efectuó en función del \extranjerismo{sprint} del momento,
  concentrando los esfuerzos en la meta fijada en él.

  \item Por último, citar el empleo de Git y de GitHub \ref{sec:vcs}. Su
  utilización ha evidenciado de nuevo su utilidad e idoneidad. En más de una
  ocasión ha sido valiosa la capacidad de poder recuperar el código fuente de
  una versión estable. Además, al estar disponible en línea ha permitido al
  tutor revisar y orientar sobre la evolución del desarrollo.
\end{itemize}

\subsection{Líneas de trabajo futuras}{\label{sec:lines-futuras}}
El proyecto realizado puede ser mejorado o ampliado en diversos frentes.
A continuación se describen aquellos aspectos de mayor relevancia.

\begin{itemize}
  \item A nivel de \extranjerismo{hardware} la placa de desarrollo tiene
  integrados más dispositivos de los utilizados en el proyecto con los que se
  podría tratar. Los botones físicos, el acelerómetro y magnetómetro, los
  canales de comunicación UART, SPI, USB, ADC; o el lector de tarjetas de
  memoria podrían proporcionar nuevas funcionalidades.
  
  \item En línea con lo anterior, la placa cuenta con la posibilidad de añadir
  módulos de conexión inalámbrica Wi-Fi y Bluetooth, de esta manera, resultaría
  interesante crear un SE conectado en red y móvil.
  
  \item El concepto de internet de las cosas (IoT) cada vez está más presente y
  la cantidad de dispositivos inteligentes conectados en red está aumentando
  rápidamente. Una mejora del proyecto vendría de la mano de la
  integración en la pila TCP/IP de alguno de los protocolos usados en IoT. Por
  ejemplo, MQTT o CoAP.

  \item En ese sentido, usando protocolos IoT se podría conectar el SE a la
  nube. Servicios como \href{https://azure.microsoft.com/es-es/services/iot-hub/}
  {Azure IoT Hub} permiten conectar, supervisar y administrar multitud de
  dispositivos inteligentes.

  \item Con esa orientación hacia IoT, también existen RTOS dedicado al asunto.
  \href{https://www.zephyrproject.org/}{Zephyr} es compatible con la placa K64F
  y además integra opciones de seguridad y protección no siempre presentes en
  SE actuales.

  \item También es cierto que hay medidas de seguridad que se pueden integrar
  directamente. Aprovechando que lwIP implementa el protocolo criptográfico
  Transport Layer Security (TLS) las comunicaciones podrían ser cifradas para 
  proporcionar seguridad en la transmisión de datos.

  \item Respecto a la aplicación web, aunque se ha tratado de que contase
  con un diseño web adaptable siempre se podría mejorar y reestructurar de
  acuerdo alguna de las metodologías usadas en el diseño de la experiencia del
  usuario.

  \item Además, se podrían incrementar las formas de interacción con la placa.
  Usando la interfaz de línea de comandos, un programa de ordenador con
  interfaz gráfica o mediante una aplicación para dispositivos móviles.

\end{itemize}
