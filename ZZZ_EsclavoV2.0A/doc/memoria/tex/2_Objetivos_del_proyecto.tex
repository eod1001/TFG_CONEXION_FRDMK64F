\capitulo{2}{Objetivos del proyecto}\label{ch:objetivos}

En este capítulo se detallan los objetivos que se pretenden conseguir con la
ejecución del proyecto. Se diferencian tres tipos de objetivos, los generales
que dan causa al proyecto, los técnicos inherentes al tipo de proyecto realizado
y los personales que se desean conseguir \extranjerismo{motu proprio}.

\section{Objetivos generales}\label{sec:obj_generales}
\begin{itemize}
  \item Configurar un sistema empotrado que sea capaz de conectarse en red.
  \item Dotar al sistema empotrado de diversas funciones usando algunos de los
  periféricos de los que dispone.
  \item Demostrar la ejecución correcta de las funciones implementadas.
  \item Crear una interfaz web que permita interactuar con el sistema empotrado.
\end{itemize}

\section{Objetivos técnicos}\label{sec:obj_tecnicos}
\begin{itemize}
  \item Configurar el \extranjerismo{hardware} de una placa de desarrollo
  FRDM-K64F para poder conectarla a través de su puerto Ethernet a una LAN.
  \item Utilizar componentes \extranjerismo{hardware} de la placa como 
  son sus LED de colores para mostrar que es capaz de comunicarse via TCP/IP.
  \item Utilizar el bus serie de datos I\textsuperscript{2}C presente en la
  placa para extender su funcionalidad y que pueda mostrar mensajes en una
  pantalla.
  \item Emplear modulación por ancho de pulsos para regular la intensidad del
  brillo de unos LED presentes en una placa de expansión.
  \item A nivel de comunicaciones, usar la implementación ligera de 
  TCP/IP ``lwIP'' para manejar las transmisiones.
  \item Crear una aplicación web usando la tecnología JSF capaz de comunicarse
  con una placa conectada en red.
\end{itemize}

\section{Objetivos personales}\label{sec:obj_personales}
\begin{itemize}
  \item Ampliar los conocimientos y la experiencia en el desarrollo de
  \extranjerismo{software} para sistemas empotrados.
  \item Conocer el conjunto de herramientas de trabajo que proporciona el
  fabricante de la placa de desarrollo.
  \item Extender el entendimiento de la familia de protocolos TCP/IP.
  \item Revisar el proceso de desarrollo de una aplicación web.
  \item Emplear el sistema de control de versiones distribuido Git a través
  de la plataforma de desarrollo GitHub.
  \item Emplear la técnica de desarrollo ágil Scrum en las diferentes fases del
  desarrollo.
  \item Aprender a usar el sistema de composición de textos \LaTeX\ y utilizarlo
  para realizar la documentación del proyecto.
  \item Aplicar las competencias adquiridas a lo largo de las diferentes
  asignaturas que componen el Máster Universitario en Ingeniería Informática.
\end{itemize}
