\capitulo{6}{Trabajos relacionados}{\label{ch:relacionados}}
En el campo de los SE se han realizado sistemas complejos y de gran valor en 
numerosos ámbitos. En este sentido, \href{https://www.nxp.com/products/processors-and-microcontrollers/arm-based-processors-and-mcus/kinetis-cortex-m-mcus/k-seriesperformancem4/k6x-ethernet/kinetis-k64-120-mhz-256kb-sram-microcontrollers-mcus-based-on-arm-cortex-m4-core:K64_120}
{NXP} muestra varias aplicaciones de los SE en las que se pueden emplear el
mismo MCU, Kinetis~\textsuperscript{\tiny\textregistered} K64, existente en la
placa de desarrollo.

Los sistemas expuestos muestran características similares a las presentes en la
placa K64F y a las usadas en este proyecto. El uso de conectividad en red via
TCP/IP hace posible la comunicación con el SE a distancia. Y la utilización de
GPIO, I\textsuperscript{2}C o PWM permite operar con los sensores, actuadores u
otros componentes del SE.

A continuación se muestran cuatro sistemas pertenecientes a diferentes sectores
de aplicación. Los ejemplos pertenecen al sector industrial, sanitario y
comercial. En concreto, los sistemas operados por SE son los siguientes:

\begin{itemize}
  \item Vehículo no tripulado
  \item Sistema de telesalud
  \item Cama de hospital eléctrica
  \item Impresora de terminal punto de venta
\end{itemize}

La funcionalidad, desempeño y ámbito de uso de las aplicaciones resultan muy
diversas. Así pues, queda patente la versatilidad y el amplio espectro de
entornos en los que se pueden encontrar SE.

\subsection{Vehículo no tripulado}{\label{sec:uv}}
Un vehículo no tripulado es aquel vehículo capaz de funcionar sin un humano a
bordo. De forma general los vehículos no tripulados se clasifican en función del
medio en el que operan. Pueden ser terrestres, aéreos, acuáticos, submarinos,
espaciales... Independientemente del tipo que sean y de que estén operados por
control remoto o sean autónomos, los vehículos se sirven de SE para controlar
sus funciones.

\imagen{uv}{Diagrama de bloques de un vehículo no tripulado \cite{webpage:uv}}

Con LED controlados por GPIO se puede conocer el estado del vehículo. Para
dotar de conectividad al vehículo están disponibles interfaces cableadas e
inalámbricas.

El uso de PWM es una característica compartida con el proyecto. Mientras que 
en el proyecto se ha usado para regular la intensidad del brillo de unas
luces, en los vehículos no tripulados se emplea PWM para regular sus motores

Aunque no se ha incluido en el proyecto, la capacidad de usar tarjetas de
memoria, de dispositivos conectados a un puerto USB o la comunicación con
periféricos usando SPI está presente tanto en vehículos como en la placa de
desarrollo.

\subsection{Sistema de telesalud}{\label{sec:salud}}
El acceso a la asistencia sanitaria no siempre es factible o asequible. Vivir
en entornos rurales alejados de centros médicos, la falta de medios de
transporte, la pérdida de movilidad de los pacientes, o la falta de recursos
económicos son causas que pueden impedir dicha asistencia. Con el avance de la
tecnología y las telecomunicaciones ha surgido la posibilidad de abordar dicha
problemática con el uso de sistemas de telesalud.

Con el uso de sensores de diverso tipo se recogen datos sobre el estado de salud
del paciente. Su presión arterial, su ritmo cardíaco, su temperatura corporal,
etc. Los datos son recogido por el sistema, procesados y retransmitidos al
personal médico encargado de velar por la salud del paciente.

En la figura \ref{fig:telesalud} se muestran los componentes que integran el
sistema. Entorno al MCU se hallan presentes una interfaz infrarroja que recoge
los datos de los sensores y una pantalla que muestra información al paciente.
También se puede ver como el sistema usa PWM para generar sonidos.

Periódicamente el sistema envía los datos bien usando una conexión Ethernet o
bien usando una conexión USB a un ordenador conectado a Internet. De manera 
opcional se puede conectar el sistema a Internet de forma inalámbrica. Este
hecho serviría para aumentar la portabilidad del sistema de telesalud al
completo.

\imagen{telesalud}{Diagrama de bloques de un sistema de telesalud
\cite{webpage:telehealth}}

\subsection{Cama de hospital eléctrica}{\label{sec:cama}}
Siguiendo en el sector sanitario, se presenta otra muestra de aplicación de SE
en las camas de hospital eléctricas. El uso de SE permite ofrecer el 
máximo confort posible al paciente y mejorar atención sanitaria recibida.

Con una serie de elementos de regulación se puede acomodar la cama a las
necesidades del usuario. Para conseguir tal comodidad el SE cuenta con motores
encargados de regular la inclinación o altura de varias partes de la cama. 

Al integrar un monitor de constantes vitales o una bomba de infusión, 
dispostivos también creados con SE, se puede realizar una monitorización y 
atención completa del paciente. Y si además se conecta la cama a una red
inalámbrica se abre la posibilidad de monitorizar y atender remotamente al
paciente. Por ejemplo, desde un control de enfermería.

Para comunicarse telefónicamente con el enfermo se utiliza la conexión
Ethernet en conjunción con la tecnología voz sobre protocolo de internet (VoIP).

Por último, que la cama disponga de una pantalla LCD permite que conocer 
\extranjerismo{in situ} el estado de todo el sistema, paciente incluido.

\imagen{cama}{Diagrama de bloques de una cama de hospital eléctrica
\cite{webpage:bed}}

\subsection{Impresora de terminal punto de venta (TPV)}{\label{sec:tpv}}
Para terminar, se pasa al ámbito del sector minorista. Se muestra el sistema
utilizado en una impresora de un TPV. La impresora está compuesta por
dos SE. Mientras que uno de ellos se encarga de las entradas digitales, el
otro se ocupa del control de los motores y dispositivos de impresión.

Las opciones de conectividad permiten comunicarse con la impresora de varias
maneras. Con la conexión Ethernet se puede trabajar en red con la impresora.
Igualmente, se pueden usar las conexión USB o RS-485.

Para controlar el \extranjerismo{hardware} relativo al proceso de impresión
se cuenta con la ayuda de un segundo SE. Como el SE auxiliar recibe señales
analógicas requiere usar ADC, el resto de entradas y salidas se conectan
directamente a los pines GPIO. Finalmente, la comunicación con el SE principal
se realiza mediante I\textsuperscript{2}C o SPI.

Por otra parte, el SE principal cuenta con mayor número de conexiones al
recibir las entradas del usuario, los controles digitales, gestionar la
conectividad y manejar los sensores.

De nuevo están presentes características utilizadas a su vez en el proyecto:
conectividad en red a través de un puerto Ethernet, conexión con periféricos
usando el bus de datos serie I\textsuperscript{2}C y comunicaciones diversas via
GPIO.

\imagen{tpv}{Diagrama de bloques de una impresora de TPV\cite{webpage:tpv}}
