\capitulo{1}{Introducción}\label{ch:introduccion}
Los sistemas empotrados o embebidos (SE) son sistemas diseñados para realiza una
función específica y, por tanto, solo realizan una o unas pocas tareas
concretas. Este hecho les diferencia de otros sistemas de propósito general como
ordenadores o teléfonos móviles, que son capaces de realizar multitud de tares
de diferente naturaleza. 

Como el propio nombre de los SE indica, este tipo de sistemas se suele encontrar
integrado en otros sistemas eléctricos o mecánicos de mayor envergadura y que se
encargan de controlar. Por esta razón, se requiere que los SE cuenten con 
ciertas características: tamaño reducido, bajo consumo o bajo coste. Requisitos
que a su vez provocan nuevos atributos, uno de ellos es menor potencia de
cálculo en comparación con otro tipo de sistemas. Además, en función de las
condiciones ambientales donde se encuentre el SE es posible que necesite ser
dotado de protección adicional ante situaciones especiales de temperatura,
humedad, vibración, etc.

Ciertos SE controlan el funcionamiento de otros sistemas que requieren que sus
operaciones se realicen de manera determinista. Es decir, una instrucción dada
se tiene que realizar de manera inmediata o con un retardo mínimo y conocido de
antemano. Para dar soporte a operaciones en tiempo real, los SE cuentan con
Sistemas Operativos en Tiempo Real (RTOS) que se encargan de repartir el tiempo
de ejecución de cada tarea, asignándolo en función de la prioridad de cada
tarea.

A nivel de \extranjerismo{hardware}, un microcontrolador es quien se encarga de
ejecutar las instrucciones programadas. Cada microcontrolador cuenta con un
microprocesador, con memoria y con determinados periféricos de entrada y salida.
Para que el SE se pueda comunicar con sus periféricos o con otros dispositivos
se emplean los buses serie de datos. Algunos de los buses más utilizados son
Inter-Integrated Circuit (I\textsuperscript{2}C), Serial Peripheral Interface
(SPI), o Universal Serial Bus (USB).

Otro medio que emplean los SE para comunicarse son las redes de comunicaciones
de datos a las que se puede conectar tanto por cable como de manera
inalámbrica. Existen varios protocolos para transmitir datos en red, un conjunto
muy común y conocido es la familia de protocolos de internet, también
conocido como pila o conjunto de protocolos TCP/IP
(Transmission Control Protocol / Internet Protocol). Esta familia de protocolos
es ampliamente utilizada para comunicar equipos en redes de área local (LAN), en
redes de área local inalámbricas (WLAN) y en todo el mundo a través de Internet.

Con el uso de TCP/IP se abre un abanico de nuevas funcionalidades permitiendo
realizar funciones, apoyadas en otros protocolos de la pila TCP/IP, que antes
no eran posibles. Por ejemplo, usando el Hypertext Transfer Protocol (HTTP) se
pueden transferir páginas web, con File Transfer Protocol (FTP) transferir
archivos, con Simple Mail Transfer Protocol (SMTP) enviar correos electrónicos
o con Message Queuing Telemetry Transport (MQTT) se pueden enviar
mensajes bajo el patrón de mensajería de publicar-suscribir.

En este trabajo se muestra como crear un SE conectado usando TCP/IP en una placa
de desarrollo FRDM-K64F del fabricante NXP. Se parte de la configuración de los
componentes \extranjerismo{hardware} y \extranjerismo{software} necesarios, para
terminar demostrando la interacción, de manera remota, con el SE desde una
aplicación web. Desde dicha aplicación es posible enviar comandos para realizar
alguna de las funciones programadas en la placa.

\section{Estructura de la memoria}\label{sec:estructura}
La presente memoria se estructura de la siguiente manera:

\begin{itemize}
\item
  \textbf{Introducción:} descripción abreviada de los temas principales 
  abordados en el proyecto. La introducción está acompañada de la estructura
  que toma la memoria y un listado con el contenido adjunto a la misma.
\item
  \textbf{Objetivos del proyecto:} declaración de los objetivos generales,
  técnicos y personales que se pretenden conseguir con el desarrollo de este
  trabajo.
\item
  \textbf{Conceptos teóricos:} explicación de los conceptos teóricos más
  relevantes en la realización del proyecto. Se tratan tanto los sistemas
  empotrados como la familia de protocolos TCP/IP.
\item
  \textbf{Técnicas y herramientas:} descripción de las técnicas y las
  herramientas que han sido empleadas para el desarrollo del proyecto. También
  se muestran las alternativas valoradas y los motivos de escoger la opción
  seleccionada.
\item
  \textbf{Aspectos relevantes del desarrollo:} presentación de los aspectos o
  facetas que han tomado mayor relevancia durante la ejecución de trabajo.
\item
  \textbf{Trabajos relacionados:} estado de la técnica en la creación de
  sistemas que emplean sistemas empotrados conectados en red.
\item
  \textbf{Conclusiones y líneas de trabajo futuras:} conclusiones extraídas tras
  la realización del proyecto, así como nuevas líneas de trabajo sobre las que
  mejorar o ampliar lo presentando en este proyecto.
\end{itemize}

\section{Anexos a la memoria}\label{sec:anexos}
La memoria se presenta acompañada de los siguientes anexos:

\begin{itemize}
\item
  \textbf{Plan del proyecto software:} exposición de la
  planificación temporal y del estudio de viabilidad del proyecto, tanto la
  económica como la legal.
\item
  \textbf{Especificación de requisitos del software:}
  presentación del catálogo de requisitos, así como la descripción de cada uno
  de ellos.
\item
  \textbf{Especificación de diseño:} descripción de la fase de diseño, el
  diseño de datos, el diseño procedimental y el diseño arquitectónico, del
  \extranjerismo{software} del SE y de la aplicación web.
\item
  \textbf{Manual del programador:} explicación en detalle de aquellos aspectos
  que un programador debe conocer para trabajar con el código fuente del
  proyecto.
\item
  \textbf{Manual de usuario:} explicación para que un usuario interesado en el
  proyecto sea capaz de instalar, configurar y operar con el SE y con la
  aplicación web.
\end{itemize}

\clearpage

\section{Contenido adjunto}\label{sec:adjunto}
Se adjunta el siguiente contenido a la memoria y los anexos:

\begin{itemize}
\item
  \extranjerismo{Software} para la placa de desarrollo \cite{webpage:repo-se}.
\item
  Aplicación web para interacción remota \cite{webpage:repo-aw}.
\item	
  Documentación del \extranjerismo{software} \cite{webpage:repo-se-doc}.
\item	
  Documentación de la aplicación web \cite{webpage:repo-aw-doc}.
\item
  Repositorio en línea con el código del \extranjerismo{software} \cite{webpage:repo-se}.
\item
  Repositorio en línea con el código de la aplicación web \cite{webpage:repo-aw}.
\end{itemize}
