\apendice{Documentación de usuario} \label{ch:man-user}



\section{Introducción} \label{sec:man-user-intro}
Para que cualquier persona interesada pueda usar el sistema desarrollado,
este apéndice muestra los pasos necesarios.

Se indican los requisitos para la puesta en marcha del sistema empotrado
y de la aplicación web. También se indican las instrucciones para su correcta
instalación. Por último, un pequeño manual explica brevemente las funciones
accesibles desde la aplicación web.



\section{Requisitos de usuarios} \label{sec:man-user-req}
Como el sistema está compuesto por el sistema empotrado y por la aplicación web
es requisito fundamental tener ambos componentes listos para su utilización
por parte del usuario.


\subsection{Requisitos del sistema empotrado} \label{sec:man-user-req-se}
El sistema empotrado debe contar con el \sw{} almacenado en su memoria. En caso
de no estarlo, se deben seguir los pasos descritos en el apéndice con la
Documentación técnica de programación \ref{ch:man-dev}, en concreto, la sección
dedicada a la Compilación, escritura y ejecución del sistema empotrado
\ref{sec:exe-se}.


Para que el sistema empotrado pueda solicitar y recibir una dirección IP se
utiliza el protocolo DHCP. Por lo tanto, se requiere de la presencia de un
servidor DHCP en la red. Habitualmente, los \extranjerismo{routers} usados
en redes pequeñas de tipo residenciales o Small Offices/Home Offices (SOHO)
llevan incluido un servidor DHCP facilitando la instalación al usuario.


\subsection{Requisitos de la aplicación web} \label{sec:man-user-req-aw}
La aplicación web es accesible desde cualquier navegador web. Se puede usar
cualquiera de los dispositivos habituales usados para acceder a la web. Bien
sean equipos de sobremesa o dispositivos móviles.

Como la aplicación se ejecuta en un servidor de aplicaciones, es necesario
contar con uno en la misma red donde se conecta el sistema empotrado. Las
instrucciones sobre como desplegar la aplicación se detallan en el apéndice
con la Documentación técnica de programación \ref{ch:man-dev}, en concreto, la
sección dedicada a la Compilación, escritura y ejecución de la aplicación web
\ref{sec:exe-aw}.



\section{Instalación} \label{sec:man-user-inst}
Si se cumplen los requisitos anteriores, la instalación se reduce a conectar y
arrancar el sistema empotrado.

Para establecer la conexión a la red basta con un cable de par trenzado
conectado al sistema empotrado en un extremo y a un \extranjerismo{switch},
\extranjerismo{router} u otro dispositivo de acceso a la red en el
otro.

Para finalizar, el arranque del sistema empotrado se realiza automáticamente
cuando se conecta un cable USB en alguno de sus puertos Micro-USB. Esta conexión
se encarga de suministrar la alimentación eléctrica al sistema.

\imagenancho{boot}{Arranque del sistema empotrado}{!h}{0.6}

Una vez encendido, el sistema empotrado solicitará una dirección IP. Tras
obtenerla, la mostrará por pantalla junto al puerto TCP abierto.
Cuando se muestran estos datos ya es posible enviar instrucciones al sistema
empotrado desde la aplicación web.



\section{Manual del usuario} \label{sec:man-user}
La aplicación web está accesible desde la URL http://servidor:8080/web-app/,
siendo ``servidor'' la dirección del servidor de aplicaciones.

\imagenancho{landing}{Vista general de la web}{!h}{0.9}

Los botones de la barra de navegación permiten desplazarse directamente a las
funciones.

\imagenancho{navigation}{Barra de navegación}{!h}{0.75}

Para poder transmitir las instrucciones al sistema empotrado hay que indicar
los ajustes de red. Un texto provisional muestra un ejemplo del tipo de
direcciones habitual en redes locales y el puerto TCP usado durante el
desarrollo.

\imagenancho{network}{Ajustes de red sin establecer}{!h}{0.75}

Estos datos los proporciona el sistema empotrado tras su arranque. 

\imagenancho{dhcp}{IP y puerto del sistema empotrado}{!h}{0.6}

Tras escribir los datos, pulsando conectar la aplicación web queda preparada
para comunicarse con ese sistema empotrado.

\imagenancho{network2}{Ajustes de red establecidos}{!h}{0.75}

La primera función disponible es el encendido de las luces de colores usando
los LED RGB. El color de cada botón refleja el color de la luz a iluminar.
Por otro lado, pulsando el botón negro se apagan todos los LED encendidos.

\imagenancho{rgb}{Colores iluminables con los LED RGB}{!h}{0.75}

La segunda función muestra cadenas de caracteres en la pantalla LCD del sistema
empotrado. Para cada línea del LCD se proporciona un cuadro de texto diferente.
Cada cuadro se acompaña de un botón para confirmar el envío del mensaje.

\imagenancho{msg}{Envío de mensajes a la pantalla LCD}{!h}{0.75}

Por ejemplo, enviando el texto provisional que aparece por defecto la pantalla
del sistema empotrado muestra lo siguiente \footnote{Como el texto provisional
supera el límite de 16 caracteres del LCD, los últimos caracteres se descartan.}

\imagenancho{msg-lcd}{Mensajes enviados a la pantalla LCD}{!h}{0.6}

La tercera y última función disponible es la regulación de la intensidad del
brillo de unos LED mediante PWM. Para realizar esta regulación se muestran 
cuatro controles deslizantes de colores. Cada uno con el color del LED que
regula.

\imagenancho{pwm}{Controles deslizantes para los LED PWM}{!h}{0.8}

Todos los controles parten de la posición inicial de apagado (o 0\%).
Deslizándolos a derecha e izquierda se puede ajustar al valor que se quiera.

\imagenancho{pwm2}{Controles ajustados a diferentes valores}{H}{0.8}
