\capitulo{4}{Técnicas y herramientas}{\label{ch:tecyherra}}
Este capítulo presenta las técnicas metodológicas y las herramientas de
desarrollo usadas durante la realización del proyecto. En algunas de las
técnicas y herramientas existen varias alternativas a utilizar. En este capítulo
se comentan las características principales de cada alternativa y se expone el
porqué de la elección tomada.


\section{Técnicas metodológicas}{\label{sec:tecnicas}}

\subsection{Scrum}{\label{sec:scrum}}
Scrum es definido por dos de sus mayores promotores \textcite{schwaber2017}
como:
\begin{quotation}``Un marco de trabajo en el que las personas pueden
  abordar problemas flexibles y complejos, mientras se entregan productos creativa y
  productivamente del mayor valor posible.''
\end{quotation}

Desde el punto de vista del desarrollo del \extranjerismo{software}, trabajar en
un producto en toda su extensión puede atraer problemas debidos, por ejemplo, a
cambios en los requisitos de los clientes. Para solventar este problema, Scrum
enuncia que el trabajo se realice de forma ligera, iterativa e incremental,
permitiendo que el desarrollo pueda responder mejor ante imprevistos.

Scrum propone grupos de trabajo donde sus integrantes tienen diferentes roles y
responsabilidades. El grupo sigue un flujo de trabajo que gira entorno al
concepto de \extranjerismo{sprint}, definido como la unidad básica de
desarrollo. Antes de cada \extranjerismo{sprint} se planifica su duración
temporal, se definen la tareas a realizar en él y cuando ya está en marcha se
revisa diariamente para analizar su evolución. Una vez terminado se comprueban
los resultados obtenidos y se proporcionan los entregables generados.

Por tanto, Scrum se ha escogido con el propósito de realizar el desarrollo de
forma ágil.


\section{Herramientas \extranjerismo{hardware} de desarrollo}
  {\label{sec:herramientas}}

\subsection{Placa de desarrollo FRDM-K64F}{\label{sec:k64f}}
El uso de placas de desarrollo sirve como toma de contacto al microprocesador y
al resto del \extranjerismo{hardware} que componen un SE. En este proyecto la
placa utilizada es el modelo
\href{https://www.nxp.com/support/developer-resources/evaluation-and-development-boards/freedom-development-boards/mcu-boards/freedom-development-platform-for-kinetis-k64-k63-and-k24-mcus:FRDM-K64F}
{FRDM-K64F}.

\imagen{k64f}{Placa de desarrollo FRDM-K64F \cite{webpage:k64f}}

El microcontrolador (MCU) de esta placa es el \href{https://www.nxp.com/products/processors-and-microcontrollers/arm-based-processors-and-mcus/kinetis-cortex-m-mcus/k-seriesperformancem4/k6x-ethernet/kinetis-k64-120-mhz-256kb-sram-microcontrollers-mcus-based-on-arm-cortex-m4-core:K64_120}
{Kinetis\textsuperscript{\tiny\textregistered} K64}. Este MCU cuenta con el
microprocesador \href{https://developer.arm.com/products/processors/cortex-m/cortex-m4}
{Cortex\textsuperscript{\tiny\textregistered}-M4} capaz de funcionar a 120 Mhz.
Destacan la capacidad de funcionar en modos de energía ultra bajos, su elevado
rendimiento y los dispositivos integrados de conectividad y comunicación como
los puertos General Purpose Input/Output (GPIO), la interfaz Ethernet o los
buses de datos serie I\textsuperscript{2}C o SPI.

La placa K64F ofrece el \extranjerismo{hardware} necesario para poder aprovechar
las características que ofrece el MCU. Por ejemplo, cuenta con los pines
suficientes para dar cabida a todos los GPIO. Además, la disposición de los
conectores es compatible con la disposición utilizada en placas
\href{https://www.arduino.cc/}{Arduino} permitiendo el uso de multitud de placas
de extensión ya existentes.

También presenta un puerto Ethernet que permite conectar la placa a una red
local y, a la postre, a Internet. También cuenta con \extranjerismo{hardware}
extra que posibilita nuevas opciones a la hora de investigar y desarrollar con
la placa. Por ejemplo, puertos USB, para alimentación, depuración y conexión; y
diodos emisor de luz (LED) de colores rojo, verde y azul (RGB).

\subsection{Placa de expansión Arduino Basic I/O}{\label{sec:basic-io}}
Como la placa K64F es compatible con las placas de expansión de Arduino se
utiliza la placa \href{https://web.archive.org/web/20160818213905/http://www.msebilbao.com/tienda/product_info.php?cPath=130&products_id=793&osCsid=f967e6ddeaaa2f19050972ff62295a08}
{Arduino Basic I/O} para poder utilizar \extranjerismo{hardware} adicional.

\imagen{basic_io}{Placa de expansión Arduino Basic I/O \cite{webpage:basio-io}}

En concreto, se utilizan sus 4 LED de colores en combinación con la técnica de
modulación de señales pulse-width modulation (PWM). Con esta técnica se consigue
regular la intensidad del brillo de cada uno de los LED.

\subsection{Pantalla de cristal líquido (LCD)}{\label{sec:lcd}}
El disponer de una pantalla permite mostrar información en forma de texto a un
usuario del SE. En particular, la pantalla utilizada puede mostrar 2 líneas de
texto de 16 caracteres cada una. Gracias a esta pantalla el usuario puede
conocer ciertos datos acerca del estado de inicialización de la placa o puede
usarla para mostrar sus propios mensajes.

Cabe destacar que la pantalla usa una conexión de tipo paralela, necesitando
hasta 16 pines para la transmisión de datos, la alimentación, la iluminación y
el control de la transmisión. Para simplificar su uso la pantalla incorpora un
módulo de comunicación I\textsuperscript{2}C reduciendo las conexiones
necesarias a solo de 4.

\subsection{Accesorios extra}{\label{sec:extras}}
Para realizar el montaje de las placas y la pantalla se utilizan varios
accesorios extra.

Las placas de pruebas \cite{webpage:placa-pruebas} son unos tableros con
orificios que permiten conectar otras placas, componentes electrónicos,
conectores y cables. Su orificios están conectados eléctricamente, siguiendo un
patrón de bloques y líneas, permitiendo así la conexión entre los elementos del
sistema a montar. La rapidez y facilidad del montaje y desmontaje de los
elementos facilita el prototipado de sistemas antes de su fabricación.

Con el fin de conectar los pines de los componentes se usan varios cables
puente \cite{webpage:cable-puente}. Algunos están conectados directamente a las
placas, otros están conectados a las placas de pruebas. Para evitar confusiones
los cables son de colores, de tal manera que es posible distinguir la función
que realiza cada uno.

Por último, como el objetivo del proyecto es realizar un SE que utilice los
protocolos TCP/IP, es necesario conectar la placa K64F con un cable Ethernet
a un determinado equipamiento de acceso a la red, por ejemplo,
un \extranjerismo{switch}.

\section{Herramientas \extranjerismo{software} de desarrollo}
  {\label{sec:herramientas}}

\subsection{Entorno de desarrollo integrado (IDE)}{\label{sec:ide}}
Opciones valoradas:
\begin{description}
  \item[\href{https://www.nxp.com/support/developer-resources/evaluation-and-development-boards/freedom-development-boards/mcu-boards/kinetis-design-studio-integrated-development-environment-ide:KDS_IDE}
  {Kinetis Design Studio IDE}:] entorno multiplataforma basado en
  \extranjerismo{software} libre como Eclipse IDE o GNU Compiler Collection
  (GCC). Incorpora Processor Expert, una utilidad que permite añadir y
  configurar los componentes necesarios para un proyecto.
  \item[\href{https://www.nxp.com/support/developer-resources/software-development-tools/mcuxpresso-software-and-tools/mcuxpresso-integrated-development-environment-ide:MCUXpresso-IDE}
  {MCUXpresso IDE}:]al igual que KDS es un IDE multiplataforma basado en Eclipse
  IDE. Dispone de unas herramientas de configuración que permiten habilitar y
  configurar los pines, los relojes y los periféricos de la placa de desarrollo
  en uso.
  \item[\href{https://www.eclipse.org/downloads/packages/release/2018-12/r/eclipse-ide-enterprise-java-developers}
  {Eclipse IDE for Enterprise Java Developers}:] IDE adaptado para Java EE.
  Facilita el despliegue de aplicaciones en servidores de aplicaciones como
  GlassFish entre otros.
\end{description}

En cuanto a los IDE para el desarrollo del \extranjerismo{software} de la placa
K64F, se contemplan los dos primeros IDE. Ambos pertenecen a NXP, pero el propio
fabricante indica que Kinetis Design Studio IDE ha dejado de ser desarrollado y
que no se recomienda su uso en nuevos proyectos. Para nuevos proyectos se insta
a usar su otro entorno MCUXpresso IDE. Por lo tanto, la opción seleccionada como
entorno de desarrollo es \textbf{MCUXpresso IDE}.

Las herramientas de configuración que incorpora MCUXpresso se agrupan bajo el
nombre de Config Tools \cite{webpage:config-tools}. La herramienta Pins Tools,
sirve para configurar los pines del MCU. Clocks Tool permite configurar
gráficamente de los relojes del sistema. Y Peripherals Tool asiste en la
configuración de los periféricos disponibles en la placa de desarrollo.

Como tercer y último componente de la \extranjerismo{suite} MCUXpresso queda
citar el MCUXpresso Software Development Kit (SDK) \cite{webpage:sdk}. El kit
incluye un repertorio amplio de \extranjerismo{drivers} y
\extranjerismo{middleware} listos para funcionar en la placa de desarrollo.

Por otra parte, para el desarrollo de la aplicación web la opción utilizada es
\textbf{Eclipse IDE}. La versión especifica para Java EE permite manejar los
servidores de aplicaciones, pudiendo arrancarlos y detenerlos desde el propio
IDE, asi como pudiendo desplegar o quitar en ellos la aplicaciones en desarrollo
directamente.

\subsection{Otras herramientas utilizadas por el SE}{\label{sec:otros}}
\begin{description}
  \item[\href{https://www.freertos.org/}{FreeRTOS}:] RTOS encargado de
  planificar la ejecución de las tareas del SE. En función la configuración
  establecidas y de las prioridades asignadas, una tarea prioritaria será capaz
  de detener la ejecución del resto. El RTOS también se puede encarga del envío
  de mensajes entre tareas, mediante mensajes directos, colas o buzones.
  \item[\href{https://savannah.nongnu.org/projects/lwip/}{lwIP}:] Implementación
  liviana de la pila de protocolos TCP/IP. Permite a la placa utilizar la
  mayoría de los protocolos habituales: IPv4, IPv6, TCP, UDP, ICMP, IGMP...
\end{description}

Estas herramientas se han utilizado por su conocimiento previo más que por 
un exhaustivo examen de todas las alternativas existentes.

\subsection{Repositorio del código fuente}{\label{sec:vcs}}
Opciones valoradas:
\begin{description}
  \item[\href{https://github.com/}{GitHub}:] La versión gratuita permite la
  publicación de repositorios ilimitados, tanto públicos como privados. Ambos
  tipos de repositorios se encuentran almacenados en su nube. También permite
  un número ilimitado de colaboradores.
  \item[\href{https://bitbucket.org/}{Bitbucket}:] De forma gratuita admite
  repositorios tanto privados como públicos, permitiendo 1 GB de almacenamiento
  en los privados. El número de colaboradores está limitado a 5. Y el
  almacenamiento de los repositorios se realiza en la nube.
  \item[\href{https://about.gitlab.com/}{GitLab}:] En la versión gratuita ofrece
  repositorios privados y públicos, disponiendo los públicos de 10 GB de
  almacenamiento. Respecto al almacenamiento cuenta con dos opciones,
  repositorios almacenados en la nube o de forma local. En cualquiera de los
  casos, permite un número ilimitado de colaboradores.
\end{description}

Las tres opciones basan su funcionamiento en Git por lo que cualquiera sería
válida. Teniendo en cuenta la posibilidad de usar ZenHub para la gestión de
proyectos, el repositorio escogido es \textbf{GitHub}.

En efecto, hay dos repositorios para el código fuente del proyecto alojados en 
GitHub, uno para el código usado por la
\href{https://github.com/rpc0027/k64f-lwip}{K64F} y otro para la
\href{https://github.com/rpc0027/web-app}{aplicación web}.

\subsection{Herramientas de documentación}{\label{sec:docs}}
Opciones valoradas:
\begin{description}
  \item[\href{https://products.office.com/es-es/word}{Word}:] Procesador de
  textos por excelencia perteneciente a la \extranjerismo{suite} ofimática
  Office de Microsoft. Sus documentos son formato cerrado pero el programa
  también permite la apertura y edición de otro tipo de documentos.
  \item[\href{https://es.libreoffice.org/descubre/writer/}{Writer}:] Procesador
  de texto de código abierto que está incluido en la \extranjerismo{suite}
  ofimática LibreOffice, también de código abierto. Sus documentos son formato
  abierto, pero también es posible editar documentos en otros formatos como los
  de Word.
  \item[\href{https://www.latex-project.org/}{\LaTeX}:] Sistema de preparación de
  documentos para la producción de documentación técnica y científica. Usado de
  manera casi estándar para la creación y publicación de documentos científicos.
  \item[\href{https://help.github.com/articles/about-wikis/}{GitHub Wikis}:] 
  Espacio integrado dentro de los repositorios de GitHub para documentar el
  contenido del proyecto, permitiendo añadir información relevante para otros
  usuarios.
  \item[\href{https://code.visualstudio.com/}{Visual Studio Code}:] Editor de
   código fuente desarrollado por Microsoft. Compatible con infinidad de
   lenguajes y su funcionalidad puede ser ampliada con el uso de extensiones.
  \item[\href{http://www.xm1math.net/texmaker/}{Texmaker}:] Editor de texto
  para \LaTeX. Incluye las funcionalidades habituales y las herramientas
  necesarias para la edición y compilación de documentos. Incorpora un visor
  de archivos pdf que permite comprobar rápidamente el estado del documento en
  edición.
\end{description}

A causa de lo declarado en los objetivos personales del proyecto
\ref{sec:obj_personales}, para la documentación de memoria y anexos se escoge
\textbf{\LaTeX{}} como herramienta para la documentación. La decisión se debe a
su calidad tipográficas y por su práctica estandarización en publicaciones
técnicas y científicas.

Para la edición del texto que hay que proporcionar a \LaTeX{} se emplea
\textbf{Visual Studio Code}, usado junto a la extensión \href{https://marketplace.visualstudio.com/items?itemName=James-Yu.latex-workshop}
{LaTeX Workshop}, que añade características y utilidades relativas a \LaTeX.
Es necesario tener instalada una distribución de \LaTeX{} para poder generar los
archivos, en este caso se usa \href{https://miktex.org/}{MikTeX}.

Además, la funciones de autocompletado y detección de errores disponibles para
los lenguajes HTML y CSS hacen apto al programa para la edición de la página
web mostrada por la aplicación web.

\subsection{Herramientas de documentación del código fuente}
{\label{sec:source-docs}}
Opciones valoradas:
\begin{description}
  \item[\href{http://www.doxygen.nl/}{Doxygen}:] Herramienta considerada como
  el estándar de facto para la documentación de códigos fuente en lenguaje C++
  aunque también soporta el lenguaje C. Esta herramienta automatiza la
  extracción de los comentarios y es capaz de generar documentación en HTML
  válida para referencia en línea o documentación en \LaTeX para referencia
  fuera de línea.
  \item[\href{https://www.oracle.com/technetwork/java/javase/documentation/javadoc-137458.html}
  {Javadoc}:] Utilidad que genera documentación en formato HTML a partir de los
  comentarios incluidos en el código fuente. Viene includa con el SDK de Java 
  y puede ser ejecutada individualmente, directamente desde el IDE o desde
  otras herramientas como Maven.
\end{description}

Dada la propia finalidad de cada herramienta, para la documentación del código
de la placa K64F se usa \textbf{Doxygen} y para la documentación de la
aplicación web se emplea \textbf{Javadoc}.

\subsection{Herramientas de comunicación}{\label{sec:comms}}
Opciones valoradas:
\begin{description}
  \item[\href{https://products.office.com/es-es/outlook/email-and-calendar-software-microsoft-outlook?rtc=1}
  {Correo electrónico}:] Solución para correos electrónicos de Microsoft. 
  Disponible en el cliente web, y aplicaciones de escritorio y dispositivos 
  móviles.
  \item[\href{https://www.skype.com/es/business/}{Skype Empresarial}:] Solución
  de comunicación perteneciente a la suite ofimática Office de Microsoft.
  Permite la comunicación mediante videollamadas, llamadas de voz o mensajería
  instantánea.
\end{description}

A parte de usar correo electrónico, el empleo de \textbf{Skype Empresarial}
permite la comunicación directa entre los interesados del proyecto.

\newpage

\subsection{Herramientas de gestión de proyectos}{\label{sec:mgmt}}
Opciones valoradas:
\begin{description}
  \item[\href{https://trello.com/}{Trello}:] \extranjerismo{Software} de
  administración de proyectos que además de contar con una aplicación web cuenta
  con aplicaciones para dispositivos móviles. Funciona de manera similar a un
  tablón digital que permite organizar en tableros los diferentes proyectos del
  usuario. En cada uno de estos tableros se pueden colocar tarjetas
  representando las tareas a realizar por los participantes del proyecto.
  \item[\href{https://www.zenhub.com/}{ZenHub}:] Herramienta para la gestión
  ágil de proyectos integrada dentro de GitHub. Cuenta con tableros donde ubicar
  las tareas desde su concepción hasta su finalización. Es compatible con la
  metodología ágil Scrum y su concepto de \extranjerismo{sprint}.
\end{description}

La herramienta de gestión de proyectos empleada es \textbf{ZenHub} por su
integración con GitHub.

\newpage

\subsection{Resumen de las herramientas y tecnologías utilizadas}
{\label{sec:resumen-herra}}
La siguiente tabla muestra las herramientas y tecnologías más relevantes en cada
parte del proyecto. Como se puede ver algunas solo están involucradas en una
parte, como el lenguaje C, mientras que otras se encuentran presentes a lo largo
de todo el desarrollo, como el uso de GitHub.

\tablaSmall{Herramientas y tecnologías utilizadas}
{l c c c}{herramientas}
{\multicolumn{1}{l}{Herramientas} & firmware & web app & memoria\\}
{
  C         & X &   &  \\
  Doxygen   & X &   &  \\
  FreeRTOS  & X &   &  \\
  lwIP      & X &   &  \\
  Java      &   & X &  \\
  JSF       &   & X &  \\
  Javadoc   &   & X &  \\
  Maven     &   & X &  \\
  GlassFish &   & X &  \\
  XHTML     &   & X &  \\
  HTML5     &   & X &  \\
  CSS3      &   & X &  \\
  \LaTeX    &   &   & X\\
  MikTeX    &   &   & X\\
  VS Code   &   & X & X\\
  GitHub    & X & X & X\\
  ZenHub    & X & X & X\\
}
